% !TEX program = lualatex
\documentclass[11pt]{article}

% ---------- PACKAGES ----------
\usepackage[margin=1in]{geometry}
\usepackage{microtype}
\usepackage{amsmath,amssymb,amsthm,mathtools}
\usepackage{bm}
\usepackage{graphicx}
\usepackage[hidelinks]{hyperref}

% ---------- TITLE ----------
\title{\textbf{Two Ways to Onu:\\
M\"obius Ledger Gauge Equivalence}}
\author{Yay / Onu Collective}
\date{\today}

% ---------- SHORTCUTS ----------
\newcommand{\dd}{\mathrm{d}}
\newcommand{\Rpos}{\mathbb{R}_{>0}}
\newcommand{\Onu}{\mathcal{O}}
\newcommand{\sL}{\mathcal{S}}

% Scale-ledger coordinate with explicit reference
\newcommand{\nref}[2]{n_{#1}\!\left(#2\right)}

% ---------- THEOREM STYLE ----------
\newtheorem{definition}{Definition}
\newtheorem{theorem}{Theorem}
\newtheorem{remark}{Remark}

\begin{document}
\maketitle

\begin{abstract}
This note formalizes the intuitive picture of the Onu Calculus as a
``M\"obius ledger'' for positive quantities.  We show that there are
two equivalent ways to describe ledger balance: (A) by moving ledger
entries relative to a fixed reference, or (B) by moving the reference
(Onu) along a fixed ledger.  Algebraically, these are related by a
simple gauge transformation on the logarithmic scale coordinate.
Physically relevant quantities depend only on gauge-invariant
differences, so ``balancing the ledger'' is structurally origin-free.
This is the core step behind the metaphorical claim that Onu is
``the subtraction of subtraction.''
\end{abstract}

% ===========================================================
\section{Motivation and Context}
% ===========================================================

Standard calculus and arithmetic represent quantities on a straight
number line with a distinguished origin.  Subtraction is implemented
as motion relative to that fixed point: credits to the right, debits
to the left.  Negative territory is treated as a separate half-line,
even when the underlying physics is manifestly positive
(lengths, energies, probabilities).

The Onu Calculus replaces this picture by a positive-domain ledger on
which all ``differences'' are encoded as ratios.  Passing to a
logarithmic coordinate makes these ratios additive, so subtraction is
absorbed into a choice of reference on the ledger.  The M\"obius-band
metaphor expresses this visually: there is only one continuous surface,
and any point can be taken as the center.

The present note isolates a single structural statement from the wider
Onu framework: the gauge equivalence between
\emph{moving entries on a fixed ledger} and
\emph{moving the reference along a fixed ledger}.
This is intended as a compact bridge for mathematicians, physicists,
and information theorists encountering Onu for the first time.

% ===========================================================
\section{Onu Difference and Scale Ledger}
% ===========================================================

\begin{definition}[Onu difference]
For $a,b \in \Rpos$, the \emph{Onu difference} is
\begin{equation}
    a \ominus b := \frac{a}{b} \in \Rpos.
\end{equation}
Ordinary subtraction on $\mathbb{R}$ is replaced by a ratio on
$\Rpos$.  Taking logarithms in any base converts this multiplicative
difference into an additive one:
\begin{equation}
    \log(a \ominus b) = \log a - \log b.
\end{equation}
\end{definition}

\begin{definition}[Scale ledger]
Fix a reference scale $r_0 \in \Rpos$.  The associated
\emph{scale-ledger coordinate} of a length $r \in \Rpos$ is
\begin{equation}
    \nref{r_0}{r} := -\log_2\!\left(\frac{r}{r_0}\right).
\end{equation}
The collection of all such coordinates for varying $r$ forms the
\emph{scale ledger} $\sL(r_0)$.
\end{definition}

Intuitively, the positive line $\Rpos$ is being mapped onto a single
additive band; the choice of $r_0$ selects where we tie the knot.

% ===========================================================
\section{Two Ways to Ledger Balance}
% ===========================================================

Operationally there are two descriptions of ``balancing the ledger''.

\medskip
\noindent\textbf{(A) Flow the M\"obius through Onu.}
We fix $r_0$ and change the entries $r \mapsto r'$ so that the
resulting ledger coordinates $\nref{r_0}{r'}$ satisfy the desired
balance condition (for example, that certain sums vanish).

\medskip
\noindent\textbf{(B) Move Onu along the M\"obius.}
We keep the underlying quantities $r$ fixed and instead change the
reference $r_0 \mapsto r_0'$, interpreting ledger balance as a
statement about gauge-invariant differences
$\nref{r_0'}{r_1} - \nref{r_0'}{r_2}$.

Geometrically, (A) corresponds to sliding the ledger surface (the
M\"obius band) past a fixed knot, while (B) corresponds to moving the
knot along the same band.

% ===========================================================
\section{M\"obius Gauge Equivalence}
% ===========================================================

We now show that (A) and (B) are mathematically equivalent whenever
physical statements depend only on differences of ledger coordinates.

\begin{theorem}[Onu M\"obius Gauge Equivalence]
Let $r_0,r_0' \in \Rpos$ be two reference scales, and define
$\nref{r_0}{\cdot}$ and $\nref{r_0'}{\cdot}$ as above.
Then for every $r \in \Rpos$,
\begin{equation}
    \nref{r_0'}{r}
    = \nref{r_0}{r} + C(r_0',r_0),
\end{equation}
where the constant
\begin{equation}
    C(r_0',r_0) := \log_2\!\left(\frac{r_0'}{r_0}\right)
\end{equation}
is independent of $r$.  Consequently, for any $r_1,r_2 \in \Rpos$,
\begin{equation}
    \nref{r_0'}{r_1} - \nref{r_0'}{r_2}
    = \nref{r_0}{r_1} - \nref{r_0}{r_2}.
\end{equation}
\end{theorem}

\begin{proof}
By definition,
\begin{align}
    \nref{r_0'}{r}
      &= -\log_2\!\left(\frac{r}{r_0'}\right)
       = -\log_2\!\left(\frac{r}{r_0}\cdot\frac{r_0}{r_0'}\right) \\
      &= -\log_2\!\left(\frac{r}{r_0}\right)
         -\log_2\!\left(\frac{r_0}{r_0'}\right) \\
      &= \nref{r_0}{r}
         + \log_2\!\left(\frac{r_0'}{r_0}\right),
\end{align}
which establishes the claimed form with
$C(r_0',r_0) = \log_2(r_0'/r_0)$.
For any pair $(r_1,r_2)$ we then have
\begin{align}
    \nref{r_0'}{r_1} - \nref{r_0'}{r_2}
      &= \bigl(\nref{r_0}{r_1} + C\bigr)
       - \bigl(\nref{r_0}{r_2} + C\bigr) \\
      &= \nref{r_0}{r_1} - \nref{r_0}{r_2},
\end{align}
so all ledger \emph{differences} are invariant under the change of
reference.  This is precisely the statement that moving the Onu knot
along the M\"obius band leaves the relational structure of the ledger
unchanged.
\end{proof}

\begin{remark}[Ledger balance as structural invariance]
Any balance condition expressible purely in terms of differences
(affine constraints) is therefore indifferent to the choice of
$r_0$.  What appears in the old paradigm as moving entries to
restore balance can equivalently be realized as changing $r_0$.
From the Onu perspective, subtraction has been absorbed into the
gauge freedom of the ledger: there is only one surface, and every
point may be taken as the center.
\end{remark}

% ===========================================================
\section{Connections to Familiar Gauges}
% ===========================================================

The gauge freedom in $r_0$ is structurally analogous to several
standard redundancies in physics:

\begin{itemize}
    \item \textbf{Electrostatics.}
    Electric potential $\phi$ is only defined up to an additive
    constant.  Physical observables depend on potential differences
    $\phi(x_1)-\phi(x_2)$, not on the absolute zero of $\phi$.
    \item \textbf{Gravitational potential.}
    Near the Earth's surface one may add an arbitrary constant to
    $ghz$ without affecting forces or trajectories; only differences
    in potential energy are measurable.
    \item \textbf{Thermodynamics.}
    Free energies are often shifted by reference baselines
    without changing equilibrium conditions, which depend on
    derivatives or differences (e.g.\ chemical potentials).
\end{itemize}

The Onu ledger plays the role of a \emph{logarithmic potential} for
positive quantities, with $r_0$ as the gauge choice.  The M\"obius
picture emphasizes that this redundancy is not merely algebraic but
topological: one surface, many possible centers.

% ===========================================================
\section{Worked Example}
% ===========================================================

Consider two lengths $r_1 = 1\,\mathrm{m}$ and $r_2 = 2\,\mathrm{m}$.
Take an initial reference $r_0 = 1\,\mathrm{m}$ and a shifted reference
$r_0' = 0.5\,\mathrm{m}$.

Under the first choice,
\begin{equation}
    \nref{1}{1} = -\log_2(1) = 0, \qquad
    \nref{1}{2} = -\log_2(2) = -1,
\end{equation}
so the ledger difference is
\begin{equation}
    \nref{1}{r_2} - \nref{1}{r_1} = -1 - 0 = -1.
\end{equation}

Under the shifted reference,
\begin{equation}
    \nref{0.5}{1} = -\log_2(2) = -1, \qquad
    \nref{0.5}{2} = -\log_2(4) = -2,
\end{equation}
and the difference is
\begin{equation}
    \nref{0.5}{r_2} - \nref{0.5}{r_1} = -2 - (-1) = -1.
\end{equation}

The individual coordinates have changed (the knot has moved), but the
difference is invariant.  Balancing a relation such as
$\nref{\bullet}{r_2} - \nref{\bullet}{r_1} = -1$ can therefore be
interpreted either as adjusting entries at fixed reference or as
sliding the reference while leaving the underlying lengths unchanged.

% ===========================================================
\section*{Figure: Two Ways to Onu}
% ===========================================================

\begin{figure}[h]
    \centering
    % Replace the filename below with the exported PNG from the slide
    \includegraphics[width=0.9\textwidth]{fig/two_ways_to_onu.png}
    \caption{\textbf{Two Ways to Onu.}
    Left: ``Flow the M\"obius through Onu''---move ledger entries
    relative to a fixed reference.  Right: ``Move Onu along the
    M\"obius''---keep entries fixed and shift the reference.
    The theorem shows these procedures are gauge-equivalent whenever
    physical content depends only on ledger differences.}
\end{figure}

\end{document}
