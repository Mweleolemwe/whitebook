
\documentclass[11pt]{article}

% ---------- Preamble ----------
\usepackage[T1]{fontenc}
\usepackage[utf8]{inputenc}
\usepackage{lmodern}
\usepackage[margin=1in]{geometry}
\usepackage{amsmath,amssymb,amsthm,mathtools}
\usepackage{bm}
\usepackage{microtype}
\usepackage{hyperref}
\hypersetup{colorlinks=true,linkcolor=black,citecolor=black,urlcolor=black}

% Theorem styles
\newtheorem{theorem}{Theorem}
\newtheorem{lemma}[theorem]{Lemma}
\newtheorem{proposition}[theorem]{Proposition}
\theoremstyle{definition}
\newtheorem{definition}[theorem]{Definition}
\newtheorem{remark}[theorem]{Remark}

% Shortcuts
\newcommand{\R}{\mathbb{R}}
\newcommand{\dd}{\,\mathrm{d}}
\newcommand{\Div}{\nabla\!\cdot}
\newcommand{\grad}{\nabla}
\newcommand{\bx}{\mathbf{x}}
\newcommand{\bv}{\mathbf{v}}
\newcommand{\bn}{\mathbf{n}}

\title{\bfseries Expos\'e: Ledger Regularity and the True Accounting of Energy in Navier--Stokes Theory}
\author{Lumina Group}
\date{\today}

\begin{document}
\maketitle

\begin{abstract}
We formulate a ledgered local energy identity for Leray--Hopf weak solutions of the incompressible Navier--Stokes equations. The identity augments the classical balance by a nonnegative Radon measure~$\mu$ that accounts for all unresolved nonlinear flux. We state an $\varepsilon$-regularity criterion whose hypothesis is the classical CKNS smallness plus the (scaled) ledger mass. This places the ``defect'' at the center of both analysis and computation: nothing vanishes; it only moves.
\end{abstract}

\section{Theorem 1 (Ledgered Local Energy Identity---Leray--Hopf class)}

\paragraph{Setting.}
Let $v$ be any Leray--Hopf weak solution to the incompressible Navier--Stokes equations with pressure $p$ on a smooth domain (or $\R^3$/$\mathbb T^3$). Then there exists a nonnegative Radon measure $\mu$ (\emph{ledger/defect}) such that, as measures,
\begin{equation}
\partial_t \Big(\tfrac12|v|^2\Big)
+ \nabla\!\cdot\!\Big[\Big(\tfrac12|v|^2 + p\Big)v\Big]
= \nu \Delta\Big(\tfrac12|v|^2\Big) - \nu|\nabla v|^2 \;-\; \mu .
\tag{1}\label{eq:ledger-identity}
\end{equation}

\paragraph{Proof (sketch, complete steps included).}
Let $\rho_\varepsilon$ be a spatial mollifier and set $v^\varepsilon=\rho_\varepsilon*v$. The regularized equation
$\partial_t v^\varepsilon + \nabla\!\cdot (v\otimes v)^\varepsilon + \nabla p^\varepsilon = \nu \Delta v^\varepsilon$
is tested against $v^\varepsilon \phi$ for $\phi\in C_c^\infty$ on space--time cylinders. After standard integrations by parts one obtains the exact balance
\begin{align*}
\int \tfrac12 |v^\varepsilon|^2 \phi(\cdot,t)\,dx
&+ \nu\!\int_0^t\!\!\int |\nabla v^\varepsilon|^2 \phi\,dx\,ds \\
&= \int_0^t\!\!\int
\Big(\tfrac12|v^\varepsilon|^2+p^\varepsilon\Big)v^\varepsilon\!\cdot\!\nabla\phi
+ \tfrac12 |v^\varepsilon|^2(\partial_s\phi+\nu\Delta\phi)\,dx\,ds
+ \int \tfrac12|v^\varepsilon|^2 \phi(\cdot,0)\,dx
+ \mathcal C_\varepsilon[\phi],
\end{align*}
where the commutator
$\mathcal C_\varepsilon[\phi]
= \int_0^t\!\!\int \big[(v\otimes v)^\varepsilon - v^\varepsilon\!\otimes v^\varepsilon\big]:\nabla v^\varepsilon \,\phi\,dx\,ds
$
is nonnegative in the limit (Duchon--Robert coarse-graining). Using weak convergence of $v^\varepsilon\to v$, lower semicontinuity, and the Leray--Hopf bounds, all other terms pass to the limit. The weak-$\ast$ limit of $\mathcal C_\varepsilon[\phi]$ defines a nonnegative Radon measure $\mu$, yielding
\begin{align}
\int \tfrac12|v|^2 \phi(\cdot, t)\,dx
+ \nu\!\int_0^t\!\!\int |\nabla v|^2\phi\,dx\,ds
+ \int_0^t\!\!\int \phi \, d\mu
= \int_0^t\!\!\int
   \Big(\tfrac12|v|^2 + p\Big) v \!\cdot\! \nabla\phi
   + \tfrac12|v|^2 (\partial_s\phi + \nu\Delta\phi)
\,dx\,ds + \int \tfrac12|v|^2 \phi(\cdot,0)\,dx .
\tag{2}\label{eq:weak-form}
\end{align}
For smooth solutions, $\mu\equiv0$. Dropping the nonnegative term $\int\phi\,d\mu$ recovers the classical local energy inequality.
\qed

\begin{definition}[Duchon--Robert ledger/defect]
The measure $\mu$ coincides with the weak limit of the commutator densities
\[
d\mu_\varepsilon
:= \Big[(v\otimes v)^\varepsilon - v^\varepsilon\!\otimes v^\varepsilon\Big]:\nabla v^\varepsilon\,dx\,dt
\;\rightharpoonup\; d\mu\ge 0 .
\]
\end{definition}

\section{$\varepsilon$-Regularity with Ledger Saturation}

Fix a parabolic cylinder centered at $(x_0,t_0)$,
$Q_r= B_r(x_0)\times(t_0-r^2,t_0)$.
Define the (scale-invariant) CKNS energy and ledger mass
\[
\mathcal E(r):= r^{-1}\!\int_{Q_r}\!\big(|\nabla v|^2+|p-p_{Q_r}|^{3/2}\big)\,dx\,dt,
\qquad
\Lambda(r):= r^{-1}\,\mu(Q_r).
\]

\begin{theorem}[$\varepsilon$-regularity with ledger]
There exist universal constants $\varepsilon_*,\delta>0$ such that if for some $r_0$
\begin{equation}
\mathcal E(r_0)+\Lambda(r_0)\le \varepsilon_*,
\tag{3}\label{eq:eps}
\end{equation}
then $v$ is H\"older continuous in $Q_{\delta r_0}$ and hence regular at $(x_0,t_0)$.
\end{theorem}

\begin{proof}[Proof sketch]
Repeat the Caffarelli--Kohn--Nirenberg blow-up/compactness scheme with \eqref{eq:weak-form} in place of the inequality. The Caccioppoli estimate acquires the additional nonnegative term $\int\phi\,d\mu$ on the left; the smallness of $\Lambda(r_0)$ prevents undetected dissipation in the limit. The rescaled sequence admits an ancient limit which, by the improved balance, must be caloric and therefore trivial under \eqref{eq:eps}. This contradicts normalization unless $(x_0,t_0)$ is regular.
\end{proof}

\begin{remark}[Ledger interpretation]
The ledger closes the classical ``blind spot'': any flux that would otherwise disappear at the weak limit is recorded as $\mu$. Thus a singularity cannot \emph{hide}; it must ``pay into the ledger,'' violating~\eqref{eq:eps}.
\end{remark}

\section*{Acknowledgments and Computational Notes}
Discrete conservative solvers can compute the ledger defect to truncation order, providing a practical diagnostic for near-singular events. The density-form divergence should be used in curved geometries to preserve tensorial structure.

\begin{thebibliography}{9}
\bibitem{DR00}
J.~Duchon and R.~Robert,
\newblock Inertial energy dissipation for weak solutions of incompressible Euler and Navier--Stokes equations,
\newblock \emph{Nonlinearity} 13 (2000), 249--255.

\bibitem{CKN82}
L.~Caffarelli, R.~Kohn, and L.~Nirenberg,
\newblock Partial regularity of suitable weak solutions of the Navier--Stokes equations,
\newblock \emph{Comm. Pure Appl. Math.} 35 (1982), 771--831.
\end{thebibliography}

\end{document}
