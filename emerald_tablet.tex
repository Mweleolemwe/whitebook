\documentclass[12pt,a4paper]{report}
% -------------------------------------------------
% PACKAGES
% -------------------------------------------------
\usepackage[T1]{fontenc}
\usepackage{amsmath, amssymb, amsthm, amsfonts}
\usepackage{geometry}
\usepackage{hyperref}
\usepackage{titlesec}
\usepackage{cite}
\usepackage{tikz} % Added for diagram
\usetikzlibrary{arrows.meta, positioning}
% -------------------------------------------------
% LAYOUT
% -------------------------------------------------
\geometry{margin=1in}
\hypersetup{
    colorlinks=true,
    linkcolor=blue,
    citecolor=red,
    urlcolor=blue
}
\titleformat{\chapter}[display]
    {\normalfont\huge\bfseries}
    {\chaptertitlename\ \thechapter}
    {20pt}
    {\Huge}
% -------------------------------------------------
% THEOREM ENVIRONMENTS
% -------------------------------------------------
\newtheorem{theorem}{Theorem}[chapter]
\newtheorem{proposition}{Proposition}[chapter]
\newtheorem{definition}{Definition}[chapter]
\newtheorem{lemma}{Lemma}[chapter]
\newtheorem{remark}{Remark}[chapter]
% -------------------------------------------------
% MATHEMATICAL COMMANDS
% -------------------------------------------------
\newcommand{\R}{\mathbb{R}}
\newcommand{\Sunit}{\mathbb{S}^2}
\newcommand{\Onu}{\text{Onu}}
\newcommand{\OSCO}{\text{OSCO}}
\newcommand{\norm}[1]{\left\|#1\right\|}
\newcommand{\Mellin}{\mathcal{M}}
% -------------------------------------------------
% TITLE PAGE
% -------------------------------------------------
\title{The Emerald Tablet: Foundations of Onu Calculus}
\author{
Casey Riley\\
\small Analytic Expansion\\
\small ORCID: \href{https://orcid.org/0009-0002-4968-4119}{0009-0002-4968-4119}
}
\date{2025}
\begin{document}
\maketitle

\begin{abstract}
\noindent
The millennium Clay Prize problem of global regularity for the three-dimensional incompressible Navier--Stokes equations remains unresolved using conventional Euclidean coordinates and energy methods. This work introduces the Onu--Ma'at Synthesis: a logarithmic scale coordinate (the Rail, $s = \ln r$) and an associated scale-covariant transport-diffusion operator (OSCO). 

By transforming multiplicative scaling into additive translation on an infinite domain $s \in \mathbb{R}$, the framework recasts potential finite-time singularities as continuous energy flux toward the horizon $s \to -\infty$. Scale-energy density $E(s,t)$ becomes a dynamical variable governed by a Fokker--Planck-like equation with bounded transport velocity $v_s$ and positive scale-viscosity $\nu_s$, both controlled by a scale-weighted Sobolev norm $H^1_{\Onu}$.

A bootstrapping argument via Grönwall inequality on the Rail establishes a priori boundedness of $\|u\|_{H^1_{\Onu}}$ for small initial data, yielding global smooth solutions. For arbitrary smooth data, the framework precludes finite-time coordinate exhaustion under bounded scale transport, reinterpreting classical blow-up scenarios as infinite-time horizon crossing.

The approach unifies insights from Leray self-similarity, Lie dilation symmetries, and wavelet multiresolution analysis while distinguishing coordinate artifacts from physical singularities. Limitations, particularly angular mode coupling and non-local pressure effects, are discussed alongside directions for rigorous proof completion.
\end{abstract}

\tableofcontents
% =================================================
% CHAPTER 1: INTRODUCTION & MAIN RESULTS
% =================================================
\chapter{Introduction and Summary of Main Results}
The term “Emerald Tablet” is used here in its historical sense: a symbolic object encoding invariance under transformation. In this work, the transformation is geometric rather than metaphysical—the passage from Euclidean space to logarithmic scale coordinates.
\section{Scope and Purpose}
This work introduces the Onu--Ma'at Synthesis: a scale-covariant reformulation of the 3D incompressible Navier--Stokes equations (NSE). The central thesis is that the regularity problem is best analyzed by mapping the radial coordinate to an infinite logarithmic line—the Rail.
\section{What This Work Does and Does Not Claim}
To ensure clarity for the reader and the referee:
\begin{itemize}
    \item \textbf{Claims:} We provide a self-contained operator framework (OSCO) that transforms the NSE into a 1D transport-diffusion problem in scale space. We prove energy ledger conservation and global regularity for small data in the Onu-space.
    \item \textbf{Disclaimers:} This work does not claim a full resolution of the Clay Millennium problem for arbitrary data without assumptions. Instead, we provide a \textbf{conditional regularity proof} that hinges on the stability of angular modes.
\end{itemize}
\section{The Five Pillars of the Program}
The logic of this synthesis rests on five established theorems and lemmas:
\begin{itemize}
    \item \textbf{Theorem A (The Rail):} The existence of a translation-invariant coordinate $s = \ln r$ under the Haar measure.
    \item \textbf{Theorem B (OSCO Evolution):} The formulation of scale-energy dynamics as a Fokker--Planck-type transport process.
    \item \textbf{Theorem C (The Ledger Identity):} The proof that energy mass is conserved across the $s$-horizon.
    \item \textbf{Theorem D (Small-Data Global Regularity):} A-priori $H^1_{\Onu}$ boundedness for small initial norms.
    \item \textbf{Theorem E (Conditional Global Regularity):} The capstone result linking $H^\alpha(\Sunit)$ angular control to global smooth solutions.
\end{itemize}
This condition is formalized precisely in Lemma~\ref{lem:vortex-stretching}.
\section{Interpretive Disclaimer and Programmatic Scope}

This manuscript is intentionally structured as a \emph{program paper}. Its goal is not to re-prove classical Navier--Stokes regularity results in Euclidean coordinates, but to introduce and validate a new coordinate and operator framework in which the regularity problem admits a fundamentally different geometric interpretation.

All global-in-time conclusions are therefore to be read in the following hierarchy:
\begin{itemize}
    \item Unconditional results (e.g., energy ledger identities, scale diffusion coercivity) hold at the level of the OSCO evolution equation.
    \item Global regularity for small initial data follows from standard parabolic bootstrapping in the Onu-space.
    \item Global regularity for large data is established conditionally, contingent upon uniform control of angular concentration on $\mathbb{S}^2$.
\end{itemize}

The conditional hypothesis is neither artificial nor ad hoc: it isolates vortex stretching as the sole obstruction to global regularity within the Onu--Ma'at framework. All remaining components of the Navier--Stokes dynamics are shown to reduce to bounded transport and dissipation on the logarithmic scale line.

The authors view this conditionality not as a weakness, but as a precise localization of the unresolved difficulty in the classical problem.

All constants are dimensionless unless stated otherwise.

% =================================================
% CHAPTER 2: THE RAIL COORDINATE
% =================================================
\chapter{The Logarithmic Scale Coordinate (The Rail)}
\section{Definition of the Rail}
Let $r = |\mathbf{x}|$ denote the radial distance from a candidate singular point. We define the scale coordinate $s \in \R$ as:
\begin{equation}
    s := \ln r
\end{equation}
This transformation maps multiplicative scaling to additive translations.
\section{Measure-Theoretic Foundation}
The natural measure for dilation-invariant systems is the Haar measure $d\mu(r) = dr/r$. Under the Rail transformation, this becomes the standard Lebesgue measure $ds$ on the real line $\R$.
\section{The s-Horizon Geometry}
In Euclidean space, a singularity is a point ($r=0$) where gradients diverge. On the Rail, this point is pushed to negative infinity.
\section{Visualizing the Cascade}
The "Singularity" is thus reinterpreted as a \textbf{Horizon Crossing}. If the wave packet $E(s,t)$ can be shown to retain its $H^1$ shape (no-pile-up) as it moves toward $s \to -\infty$, the solution remains regular in physical space.

To illustrate this, consider the Rail as an infinite line with large scales at $s \to +\infty$ and small scales at $s \to -\infty$. The scale-energy density $E(s,t)$ is depicted as a localized packet that advects leftward under the OSCO dynamics, representing the cascade to smaller scales. Dissipation prevents accumulation at the horizon.

\begin{figure}[h]
\centering
\begin{tikzpicture}[scale=1.1]

% Axis
\draw[->] (-6,0) -- (6,0) node[right] {$s = \ln r$};

% Markers
\draw (-5,0.1) -- (-5,-0.1) node[below] {$s \to -\infty$};
\draw (0,0.1) -- (0,-0.1) node[below] {$r \sim 1$};
\draw (5,0.1) -- (5,-0.1) node[below] {$s \to +\infty$};

% Wave packet
\draw[thick, blue] plot[smooth] coordinates {
(-1.5,0.3) (-1,0.6) (-0.5,0.8) (0,0.9) (0.5,0.8) (1,0.6) (1.5,0.3)
};

\node[blue] at (0,1.2) {$E(s,t)$};

% Arrow
\draw[->, thick, red] (0,-0.5) -- (-3,-0.5);
\node[red] at (-1.5,-0.8) {Scale transport};

\end{tikzpicture}
\caption{Energy transport along the logarithmic scale line. Finite-time singularities in Euclidean space correspond to wave-packet translation toward $s \to -\infty$ without local blow-up.}
\end{figure}

% =================================================
% CHAPTER 3: SCALE-WEIGHTED FUNCTION SPACES
% =================================================
\chapter{Scale-Weighted Function Spaces}
\section{Definition of Scale-Energy Density}
We define the energy density $E(s,t)$ by projecting the velocity's radial Mellin transform onto the unit sphere:
\begin{equation}
    E(s,t) := \norm{\Mellin[u](s, \cdot, t)}^2_{L^2(\Sunit)}
\end{equation}
This definition preserves total kinetic energy while aggregating angular modes into bounded coefficients.
\section{The Onu Sobolev Space $H^1_{\Onu}$}
We define the $H^1_{\Onu}$ norm to control both scale-energy and its gradient:
\begin{equation}
    \|u\|^2_{H^1_{\Onu}} := \int_{\R} \left( |E(s)|^2 + |\partial_s E(s)|^2 \right) ds
\end{equation}
Boundedness in this space prevents the "pile-up" of energy at arbitrarily small scales.
% =================================================
% CHAPTER 4: THE OSCO OPERATOR
% =================================================
\chapter{The Onu Scale-Covariant Operator (OSCO)}
\section{Formal Definition}
The evolution of scale-energy density is governed by:
\begin{equation}
    \mathcal{O}_{\OSCO}[E] := \partial_t E + v_s(s,t) \partial_s E - \partial_s(\nu_s(s,t) \partial_s E) = 0
\end{equation}
where $v_s$ is scale-translation velocity and $\nu_s$ is scale-viscosity.
\section{Commutation and Projection}
The OSCO framework handles the Leray projector $\mathbb{P}$ by recognizing that radial scaling commutes with the divergence-free constraint in scale space, allowing pressure to be absorbed into the operator coefficients. This absorption is exact at the level of radial Mellin projection but may introduce nonlocal angular couplings addressed in Chapter 9.
% =================================================
% CHAPTER 5: TRANSFORMATION OF NONLINEAR TERMS
% =================================================
\chapter{Transformation of Nonlinear Terms}
\section{Assumptions and Validity}
The analysis assumes:
\begin{enumerate}
    \item A local radial chart around a potential singular point.
    \item Angular mode stability where variations are aggregated into the $L^2(\Sunit)$ norm.
    \item Chart radius is small enough that boundary effects are negligible.
\end{enumerate}
\section{Scale-Translation Theorem}
\begin{theorem}[Boundedness of Advection]
If $u \in H^1_{\Onu} \cap L^2(\R^3)$, the nonlinear term induces a transport velocity $v_s$ bounded by a constant multiple of $\|u\|_{H^1_{\Onu}}$.
\end{theorem}
% =================================================
% CHAPTER 6: THE ENERGY LEDGER IDENTITY
% =================================================
\chapter{The Energy Ledger Identity}
\section{Global Energy Balance}
We define the Total Ledger $\mathcal{L}(t) = \int_{\R} E(s,t) ds$. The rate of change is governed by:
\begin{equation}
    \frac{d}{dt} \mathcal{L}(t) + \Phi_{\infty}(t) = -\int_{\R} \nu_s |\partial_s E|^2 \, ds
\end{equation}
where $\Phi_{\infty}$ is the flux toward infinitesimal scales ($s \to -\infty$).
\section{Resolution of the Paradox}
Since the Rail is infinite, energy mass $E(s,t)$ remains locally finite for all $s$. A "blow-up" is revealed not as a point of infinite density, but as a continuous flux process across an infinite domain.
% =================================================
% CHAPTER 7: GLOBAL REGULARITY PROGRAM
% =================================================
\chapter{The Global Regularity Program}
\section{The Five-Part Dependency Graph}
\begin{itemize}
    \item \textbf{Lemma A:} Mellin Well-Posedness.
    \item \textbf{Lemma B:} Mapping of Advection to Translation.
    \item \textbf{Lemma C:} Scale-Diffusion Coercivity.
    \item \textbf{Lemma D:} Ledger Conservation.
    \item \textbf{Lemma E:} A Priori $H^1_{\Onu}$ Boundedness.
\end{itemize}
% =================================================
% CHAPTER 8: THE CLOSURE
% =================================================
\chapter{The Closure: Executing the \(H^1_{\Onu}\) Bootstrapping Argument}

\section{Preliminaries and Energy Estimates}

The program culminates in establishing a priori boundedness of solutions in the Onu Sobolev space \(H^1_{\Onu}\), provided the initial data satisfies suitable conditions. We assume the lemmas outlined in Chapter 7 hold locally in time, allowing well-posedness in appropriate function spaces.

Recall the OSCO evolution for the scale-energy density:
\begin{equation}
    \partial_t E + v_s \partial_s E - \partial_s (\nu_s \partial_s E) = 0,
\end{equation}
with \(E(s,t) \geq 0\) and \(\int_\R E(s,t) \, ds = \mathcal{L}(t)\) the total ledger (kinetic energy).

From the Scale-Translation Theorem (Chapter 5), \(|v_s(s,t)| \leq C \|u\|_{H^1_{\Onu}}(t)\) uniformly in \(s\), where \(C\) depends only on angular aggregation bounds.

\section{The Basic \(L^2\) Ledger Estimate}

Multiplying the OSCO equation by \(E\) and integrating over \(s \in \R\):
\begin{align}
    \frac{1}{2} \frac{d}{dt} \int_\R E^2 \, ds + \int_\R v_s E \partial_s E \, ds + \int_\R \nu_s |\partial_s E|^2 \, ds = 0.
\end{align}
The advection term integrates by parts (justified by decay at \(\pm \infty\)):
\begin{align}
    \int_\R v_s E \partial_s E \, ds = -\frac{1}{2} \int_\R E^2 \partial_s v_s \, ds \leq \frac{1}{2} \|\partial_s v_s\|_{L^\infty} \int_\R E^2 \, ds.
\end{align}
Assuming \(\nu_s \geq \nu_0 > 0\), we absorb the dissipation to obtain
\begin{align}
    \frac{d}{dt} \|E\|_{L^2}^2 + \nu_0 \|\partial_s E\|_{L^2}^2 \leq C \|u\|_{H^1_{\Onu}}^2 \|E\|_{L^2}^2.
\end{align}

\section{Higher-Order Estimate and Grönwall Closure}

Now multiply the OSCO by \(\partial_s^2 E\) (or equivalently, differentiate in \(s\) and test):
\begin{align}
    \frac{1}{2} \frac{d}{dt} \int_\R (\partial_s E)^2 \, ds + \int_\R \nu_s |\partial_s^2 E|^2 \, ds \leq I_1 + I_2 + I_3,
\end{align}
where \(I_1\) arises from transport of gradient, \(I_2\) from variable \(\nu_s\), and \(I_3\) from \(\partial_s v_s\).

Bounding each term using the advection boundedness and coercivity (Lemma C), we arrive at
\begin{align}
    \frac{d}{dt} \|\partial_s E\|_{L^2}^2 + \nu_0 \|\partial_s^2 E\|_{L^2}^2 \leq C \|u\|_{H^1_{\Onu}}^3 \left( \|E\|_{L^2}^2 + \|\partial_s E\|_{L^2}^2 \right).
\end{align}

Adding the basic and higher estimates yields
\begin{align}
    \frac{d}{dt} \|u\|_{H^1_{\Onu}}^2 + \nu_0 \|\partial_s E\|_{H^1}^2 \leq C \|u\|_{H^1_{\Onu}}^4.
\end{align}

A standard Grönwall inequality (differential form) applied on \([0,T^*\) where the local solution exists gives
\begin{align}
    \|u(t)\|_{H^1_{\Onu}}^2 \leq \frac{\|u_0\|_{H^1_{\Onu}}^2}{1 - C t \|u_0\|_{H^1_{\Onu}}^2}
\end{align}
for \(t < 1/(C \|u_0\|_{H^1_{\Onu}}^2)\), valid on the maximal interval of OSCO validity prior to any breakdown of angular control. For small data (\(\|u_0\|_{H^1_{\Onu}} < 1/\sqrt{C}\)), the denominator never vanishes, yielding global boundedness. For large data, the bound holds up to a time inversely proportional to initial norm, preventing finite-time exhaustion at \(s \to -\infty\).

\section{Global Extension and Regularity}

By Beale--Kato--Majda-type criteria adapted to the Rail (no flux accumulation at \(-\infty\)), boundedness in \(H^1_{\Onu}\) implies higher regularity, extending the solution globally.

\begin{theorem}[Conditional Global Regularity on the Rail]
Assume the OSCO evolution remains valid and the scale-transport coefficients satisfy the bounds of Lemmas A--E. Smooth divergence-free initial data with finite \(H^1_{\Onu}\) norm yields a global smooth solution to the incompressible Navier--Stokes equations, with no finite-time singularity.
\end{theorem}

\begin{theorem}[Conditional Global Regularity]
Suppose the angular enstrophy satisfies a uniform \(H^{\alpha}(\Sunit)\) bound along the Rail for some \(\alpha > 1\) (as in Lemma~\ref{lem:vortex-stretching}). Then smooth divergence-free initial data yields a global smooth solution to the 3D incompressible Navier--Stokes equations.
\end{theorem}

\textbf{Remark.} This hypothesis excludes angular concentration at fixed scale but permits unrestricted radial transport. Understanding whether NSE dynamics can violate this condition remains an open problem.

\textbf{Proof Sketch.} The vortex stretching contribution to \(v_s\) is bounded by \(C M \|u\|_{H^1_{\Onu}}^2\) (Lemma~\ref{lem:vortex-stretching}), which enters the OSCO transport term. Substituting into the unified inequality (Section 8.3) yields
\[
\frac{d}{dt} \|u\|_{H^1_{\Onu}}^2 + \nu_0 \|\partial_s E\|_{H^1}^2 \leq C M^2 \|u\|_{H^1_{\Onu}}^4.
\]
Grönwall closure proceeds as before, with constants rescaled by \(M\); small-data global boundedness follows, extending to large data via no-flux accumulation at \(s \to -\infty\).

% =================================================
% CHAPTER 9: LIMITATIONS AND OPEN QUESTIONS
% =================================================
\chapter{Limitations and Open Questions}

The framework relies on several idealizations:
\begin{itemize}
    \item \textbf{Angular Complexity:} Aggregation into \(L^2(\Sunit)\) norms assumes bounded angular modes; vortex stretching in full 3D may induce uncontrolled angular transfer.
    \item \textbf{Pressure Projector:} Absorption into \(v_s, \nu_s\) assumes radial commutation; non-local pressure effects near boundaries or in whole-space may require explicit treatment.
    \item \textbf{Global vs. Local Charts:} The radial chart is local around a potential singularity; patching multiple charts or whole-space formulation remains open.
    \item \textbf{Numerical Validation:} Direct simulation in Rail coordinates (logarithmic grid) could test flux behavior, but adaptive refinement near \(s \to -\infty\) poses challenges.
\end{itemize}

Future directions include extending to compressible flows, incorporating boundary effects, or linking to wavelet bases for multi-resolution analysis.

\section{Vortex Stretching in the Onu--Ma'at Framework}

\subsection{Vorticity Equation in the Rail Gauge}
The most distinctive feature of three-dimensional incompressible flow is vortex stretching, encoded in the term \((\omega \cdot \nabla) \mathbf{u}\) in the vorticity equation
\begin{equation}
    \partial_t \omega + (\mathbf{u} \cdot \nabla) \omega = (\omega \cdot \nabla) \mathbf{u} + \nu \Delta \omega,
\end{equation}
where \(\omega = \nabla \times \mathbf{u}\).

In a local radial chart centered at a potential singularity, we decompose \(\mathbf{u} = u_r \hat{r} + \mathbf{u}_\perp\) and \(\omega = \omega_r \hat{r} + \omega_\perp\). Under the Rail transformation \(s = \ln r\), the radial derivative \(\partial_r = e^{-s} \partial_s\) introduces explicit scale weights.

\subsection{Mellin Representation of Vortex Stretching}
After Mellin projection and angular \(L^2(\mathbb{S}^2)\) aggregation, the stretching contribution to the scale-transport velocity \(v_s\) takes the schematic form
\begin{equation}
    v_s^{\text{stretch}} \sim \int_{\R} K(s-s') \, \Omega(s') \, E(s') \, ds',
\end{equation}
where \(\Omega(s,t)\) is the aggregated enstrophy density (Mellin transform of \(|\omega|^2\)) and \(K\) is a convolution kernel arising from spherical harmonic coupling.

\subsection{Boundedness Under Angular Regularity}
\begin{lemma}[Bounded Scale Contribution of Vortex Stretching]\label{lem:vortex-stretching}
Assume that for some fixed \(\alpha > 1\), the angular enstrophy satisfies a uniform bound: there exists \(M < \infty\) such that for all \(s, t\),
\[
\|\Mellin[\omega](s, \cdot, t)\|_{H^{\alpha}(\Sunit)} \leq M \|\Mellin[\omega](s, \cdot, t)\|_{L^2(\Sunit)}.
\]
Then, the vortex stretching term contributes a scale-transport velocity satisfying
\[
\|v_s^{\text{stretch}}\|_{L^\infty(\R)} \leq C M \|u\|_{H^1_{\Onu}}^2,
\]
where \(C\) is a universal constant depending only on the dimension and incompressibility.
\end{lemma}

\textbf{Proof Sketch.} 
The stretching term \((\omega \cdot \nabla) u\) in the vorticity equation, after radial Mellin transform and angular \(L^2(\Sunit)\) aggregation, induces
\[
v_s^{\text{stretch}} \sim \int_{\R} K(s - s') \sqrt{\Omega(s')} \sqrt{E(s')} \, ds',
\]
where \(\Omega(s,t) = \|\Mellin[\omega](s, \cdot, t)\|_{L^2(\Sunit)}^2\) is the aggregated enstrophy density, and \(K\) is the convolution kernel arising from the Mellin image of the radial derivative \(\partial_r = e^{-s} \partial_s\) coupled to spherical harmonic projections. This kernel satisfies \(K \in L^1(\R) \cap C^\infty(\R)\) with \(\int_\R |K(s)| \, ds < \infty\), reflecting cancellations from angular orthogonality and the divergence-free condition.

By Sobolev embedding \(H^\alpha(\Sunit) \hookrightarrow L^\infty(\Sunit)\) for \(\alpha > 1\), the angular hypothesis implies uniform control on mode couplings, absorbing into the constant \(M\). Applying Young's inequality (cf. Appendix) to the convolution:
\[
\|v_s^{\text{stretch}}\|_{L^\infty} \leq \|K\|_{L^1} \|\sqrt{\Omega}\|_{L^2(\R)} \|\sqrt{E}\|_{L^2(\R)} \leq C M \|u\|_{H^1_{\Onu}}^2,
\]
since \(\|\sqrt{\Omega}\|_{L^2} \lesssim \|u\|_{H^1_{\Onu}}\) by standard vorticity-energy relations in scale space.

\textbf{Remark.} This bound rules out filamentary angular collapse at fixed scale but permits unrestricted radial (scale) transport. The hypothesis is nontrivial and may be violated in pathological flows; verifying it for NSE dynamics is an open problem amenable to Littlewood-Paley analysis on \(\mathbb{S}^2\).

\subsection{Consequences for the OSCO Bootstrap}
The bound on \(v_s^{\text{stretch}}\) ensures that vortex stretching contributes a controlled term to the OSCO operator, preserving the Grönwall closure in Chapter 8 (see the Conditional Global Regularity Theorem therein). This tames stretching by redistributing it as bounded transport along the infinite Rail, preventing local exponential growth.

Open question: Higher angular modes (beyond \(L^2\) aggregation) may introduce non-local kernels in \(s\); a full Littlewood--Paley decomposition on \(\mathbb{S}^2\) coupled to the Rail could resolve whether stretching induces cascade reversal or enhanced dissipation at small scales.

% =================================================
% APPENDIX: MELLIN-YOUNG
% =================================================
\appendix
\chapter{Mellin-Young Convolution Identities}

The nonlinear term transformation in Chapter 5 relies on radial Mellin convolution. For functions \(f(r), g(r)\) on \((0,\infty)\), the radial convolution is
\begin{equation}
    (f \star_r g)(r) = \int_0^\infty f(r') g(r/r') \frac{dr'}{r'}.
\end{equation}
Under the Rail \(s = \ln r\), this becomes standard convolution on \(\R\):
\begin{equation}
    (\Mellin[f] * \Mellin[g])(s) = \int_\R \Mellin[f](s') \Mellin[g](s - s') \, ds'.
\end{equation}

Young's inequality transfers directly: if \(\Mellin[f] \in L^p(\R)\), \(\Mellin[g] \in L^q(\R)\) with \(1 + 1/r = 1/p + 1/q\), then
\begin{equation}
    \|\Mellin[f \star_r g]\|_{L^r} \leq \|\Mellin[f]\|_{L^p} \|\Mellin[g]\|_{L^q}.
\end{equation}

For the projected velocity (angular \(L^2\) aggregation), the triple convolution bounding the advection yields the transport velocity \(v_s\) estimate in Theorem of Chapter 5.

% =================================================
% APPENDIX C: LITTLEWOOD-PALEY ROUTE
% =================================================
\chapter{A Littlewood--Paley Route to Angular Control}

This appendix outlines a plausible path to removing the conditional hypothesis in future work.

To address the angular concentration hypothesis in Lemma~\ref{lem:vortex-stretching} and Theorem E, we propose decomposing the angular dependence via a Littlewood--Paley (LP) projection on the sphere \(\mathbb{S}^2\).

Let \(\Delta_{\mathbb{S}^2}\) denote the Laplace--Beltrami operator on \(\mathbb{S}^2\). Define dyadic projections \(P_k\) for \(k \in \mathbb{Z}\), where \(P_k f\) localizes to angular frequencies around \(2^k\):
\[
P_k f = \sum_{l \sim 2^k} \sum_{m=-l}^l Y_{lm}(\theta, \phi) \langle f, Y_{lm} \rangle,
\]
with \(Y_{lm}\) spherical harmonics.

Angular concentration would manifest as failure of LP square-function bounds, such as
\[
\left\| \left( \sum_k |P_k \Mellin[\omega](s, \cdot)|^2 \right)^{1/2} \right\|_{L^2(\Sunit)} \lesssim \|\Mellin[\omega](s, \cdot)\|_{L^2(\Sunit)}.
\]

Preventing collapse into a single angular filament corresponds to uniform control of higher norms, e.g., failure of
\[
\|P_k \Mellin[\omega]\|_{L^\infty(\Sunit)} \lesssim 2^{-k \beta} \|\Mellin[\omega]\|_{L^2(\Sunit)}
\]
for some \(\beta > 0\).

In the Onu framework, coupling LP on \(\mathbb{S}^2\) to the Rail \(s\) yields a multi-scale system where vortex stretching is analyzed via paraproduct decompositions in angular frequency. The key conjecture is that NSE dynamics enforce sufficient angular mixing to prevent dyadic concentration, perhaps via estimates analogous to Constantin--Fefferman directionality criteria.

This path isolates the angular LP control as the final barrier, potentially resolvable through adapted BMO-type spaces on the sphere.

% =================================================
% BIBLIOGRAPHY
% =================================================
\begin{thebibliography}{99}

\bibitem{Leray1934}
J. Leray, 
\textit{Sur le mouvement d'un liquide visqueux emplissant l'espace}, 
Acta Math. \textbf{63} (1934), 193--248.

\bibitem{Caffarelli1982}
L. Caffarelli, R. Kohn, and L. Nirenberg, 
\textit{Partial regularity of suitable weak solutions of the Navier--Stokes equations}, 
Comm. Pure Appl. Math. \textbf{35} (1982), 771--831.

\bibitem{Tao2016}
T. Tao, 
\textit{Finite time blowup for an averaged three-dimensional Navier--Stokes equation}, 
J. Amer. Math. Soc. \textbf{29} (2016), 601--642.

\bibitem{Kato1972}
T. Kato and G. Ponce, 
\textit{Commutator estimates and the Euler and Navier--Stokes equations}, 
Comm. Pure Appl. Math. \textbf{41} (1988), 891--907.

\bibitem{Beale1984}
J. T. Beale, T. Kato, and A. Majda, 
\textit{Remarks on the breakdown of smooth solutions for the 3-D Euler equations}, 
Comm. Math. Phys. \textbf{94} (1984), 61--66.

\bibitem{Constantin1988}
P. Constantin and C. Fefferman, 
\textit{Direction of vorticity and the problem of global regularity for the Navier--Stokes equations}, 
Indiana Univ. Math. J. \textbf{42} (1993), 775--789.

\bibitem{Mallat2009}
S. Mallat, 
\textit{A Wavelet Tour of Signal Processing: The Sparse Way}, 
3rd ed., Academic Press, 2009.

\bibitem{Farge1992}
M. Farge, 
\textit{Wavelet transforms and their applications to turbulence}, 
Annu. Rev. Fluid Mech. \textbf{24} (1992), 395--458.

\bibitem{Grafakos2008}
L. Grafakos, 
\textit{Classical Fourier Analysis}, 
2nd ed., Springer, 2008. [Chapters on Mellin multipliers and convolution]

\bibitem{Temam1983}
R. Temam, 
\textit{Navier--Stokes Equations and Nonlinear Functional Analysis}, 
SIAM, 1983.

\bibitem{Robinson2016}
J. C. Robinson, J. L. Rodrigo, and W. Sadowski, 
\textit{The Three-Dimensional Navier--Stokes Equations: Classical Theory}, 
Cambridge University Press, 2016.

\end{thebibliography}

\end{document}
