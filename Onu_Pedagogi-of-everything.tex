% !TEX program = lualatex
\documentclass[12pt]{article}

% ---------- PACKAGES ----------
\usepackage[margin=1in]{geometry}
\usepackage{amsmath,amssymb,mathtools,amsthm}
\usepackage{graphicx}
\usepackage{bm}
\usepackage{booktabs}
\usepackage{xcolor}
\usepackage[hidelinks]{hyperref}
\usepackage{titlesec}
\usepackage{enumitem}

% ---------- TYPOGRAPHY & MACROS ----------
\setlength{\parskip}{0.8em}
\setlength{\parindent}{0pt}
\newcommand{\dd}{\mathrm{d}}
\newcommand{\Onu}{\mathcal{O}}
\newcommand{\sL}{\mathcal{S}}
\newcommand{\I}{\mathcal{I}}
\newcommand{\R}{\mathbb{R}}
\newcommand{\C}{\mathbb{C}}
\newcommand{\ds}{\dd s}
\newcommand{\ellP}{\ell_{\mathrm{P}}}
\newcommand{\rg}{r_{\mathrm{g}}}

% ---------- TITLE BLOCK ----------
\title{\textbf{The Onu Calculus: A Pedagogi of Everything}\\
\large \textit{Scale, Recursion, Compactness, and the Art of Knowing}}
\author{Onu Collective}
\date{\today}

\begin{document}
\maketitle

% ===========================================================
\section*{Pedagogical Preface}
% ===========================================================
The aim of the Onu Pedagogi is not to propose new physics, but to provide a
single cognitive grammar for all existing descriptions. The triplet of
\emph{Scale, Recursion, Compactness} serves as the irreducible backbone of this
framework. By working on a positive ledger and using additive scale coordinates,
apparent theoretical boundaries dissolve into gauge choices. The result is a
unified method for teaching physics, information, and geometry as consistent
projections of one recursive structure.

\begin{abstract}
This document presents the \textbf{Onu Calculus}, a unifying cross-domain pedagogy
that reorganizes known physics into a single recursive mathematical structure. 
Rather than proposing new physical theories, Onu Calculus provides a 
\textbf{one-parameter, scale-additive, recursion-stable framework} that connects 
Newtonian, relativistic, quantum, and informational regimes through a shared 
logarithmic gauge. The goal is *knowing*, not speculation: natural structures 
emerge from consistency, compactness, and recursion. GR, QM, thermodynamics, and 
information flow appear as projections of a single ledger transform.
\end{abstract}

\tableofcontents
\newpage

% ===========================================================
\section{Axioms of the Onu Calculus}
% ===========================================================

The Onu Calculus begins from three axioms. These define structure; 
all physics-like behaviors emerge as corollaries.

\subsection{Axiom A (Scale)}
Every measurable extent $r\in \R^+$ is mapped to an additive scale index
\begin{equation}
    n(r) = -\log_2\!\left(\frac{r}{r_0}\right), 
\end{equation}
where $r_0$ is a reference length (taken as $1\,$m unless noted).
This map is bijective and generates the \emph{scale ledger} $\sL$.

\subsection{Axiom B (Recursion)}
Any quantity $X(r)$ evolving multiplicatively over $r$ evolves additively over $n$:
\begin{equation}
    X(2^{-n}) = e^{-kn} X(1),
\end{equation}
for some scale-coupling $k$.  
This is the fundamental recursion principle.

\subsection{Axiom C (Compactness)}
All physically relevant dynamics depend on a 
\textbf{single compactness parameter}
\begin{equation}
    u = \frac{2GM}{R c^2},
\end{equation}
which governs Newton, GR, redshift, diffusion, and information flow.

\subsection{The Onu Operator}
To work entirely within the positive quotient domain $\R_{>0}$, we define the
\emph{Onu difference} between two positive quantities by
\begin{equation}
    a \ominus b := \frac{a}{b}.
\end{equation}
Ordinary subtraction becomes a ratio, and logarithms map this multiplicative
calculus into standard additive form:
\begin{equation}
    \log(a \ominus b) = \log a - \log b.
\end{equation}
The Onu operator is the fundamental tool for comparing scales; both ledger
coordinates $n(r)$ and horizon coordinates $s$ are linearizations of this same
operation.

% ===========================================================
\section{Scale Ledger: Additive Physics}
% ===========================================================

The ledger transform converts extreme multiplicative hierarchies into 
finite additive structures.

\subsection{Planck–Astro Gap}
The Planck length $\ellP = 1.616\times10^{-35}$ m maps to
\begin{equation}
    n(\ellP) \approx 115.6.
\end{equation}

A neutron-star gravitational radius $\sim 10^4$ m yields
\begin{equation}
    n(\rg) \approx -12.5.
\end{equation}

Thus the entire quantum-to-relativistic universe spans
\begin{equation}
    \Delta n = 128.1 \text{ bits}.
\end{equation}
This finiteness is the basis for unification.

% ===========================================================
\section{Compactness Interfaces: Newton $\to$ GR}
% ===========================================================

\subsection{Newtonian compactness}
\begin{equation}
    u = \frac{2GM}{R c^2}, \qquad 0 < u < 1.
\end{equation}
In Newtonian regime $u \ll 1$,
\begin{equation}
    \Phi_N = -\frac{GM}{R}, \qquad v_{\rm esc} = c \sqrt{u}.
\end{equation}

\subsection{GR corrections}
\begin{align}
    g_{\rm GR}(R) &= \frac{GM}{R^2 \sqrt{1-u}}, \\
    z_{\rm GR} &= (1-u)^{-1/2} - 1.
\end{align}

\subsection{Redshift inversion}
\begin{equation}
    u = 1 - (1+z_{\rm GR})^{-2}.
\end{equation}
This is the key link between information, geometry, and recursion.

% ===========================================================
\section{Onu Information Flow}
% ===========================================================

\subsection{Ledger diffusion}
On the scale-ledger, information density $\I(n,t)$ evolves via
\begin{equation}
    \frac{\partial \I}{\partial t} = D \frac{\partial^2 \I}{\partial n^2},
\end{equation}
a 1D diffusion equation. Black-hole evaporation reduces to diffusion with a boundary condition at $n_s$, the horizon scale.

\subsection{Page curve recursion}
The Page curve becomes a triangle:
\begin{equation}
    S(t) = S_{\rm max}\,\min\!\left(\frac{t}{t_{\rm Page}}, 2 - \frac{t}{t_{\rm Page}}\right).
\end{equation}
No new physics is added; the ledger removes the coordinate singularity.

% ===========================================================
\section{Unified Horizon Recursion}
% ===========================================================

Define ledger coordinate relative to the Schwarzschild radius:
\begin{equation}
    s = \ln\left(\frac{r}{r_s}\right).
\end{equation}

Then:
\begin{itemize}
    \item $s = 0$ is the horizon,
    \item $s < 0$ interior,
    \item $s > 0$ exterior,
    \item all divergent GR functions become smooth functions on $s$.
\end{itemize}

\paragraph{Relation between $n$ and $s$.}
Using Axiom A, we may express the horizon coordinate as an affine transform of
the ledger coordinate:
\begin{equation}
    s = \ln\!\left(\frac{r}{r_s}\right)
      = \ln\!\left(\frac{r_0}{r_s}\right) - n(r)\,\ln 2.
\end{equation}
Thus $s$ and $n$ are simply two gauges on the same multiplicative structure.
All ledger-consistent statements are invariant under such gauge changes.

% ===========================================================
\section{Domain Priority Function}
% ===========================================================

Given the family of compactness parameters $u_X(R,M)$, we define the
\emph{dominant domain} at a given scale by
\begin{equation}
    D(R,M) = \arg\max_X u_X(R,M).
\end{equation}
This prescription provides a single, scale-consistent rule for selecting which
description---Newtonian, GR, or quantum-informational---is structurally
appropriate. No new dynamics are introduced: the Domain Priority Function is a
\emph{pedagogical} device that replaces ad hoc regime boundaries with a unified
criterion. In the Onu Pedagogi, this is the rule-of-thumb for choosing the
correct projection of the ledger at any scale.

% ===========================================================
\section{Unified Recursive Interpretation}
% ===========================================================

\subsection{Knowing instead of theory}
Onu Calculus does not posit new physical entities; it reorganizes experience.
\begin{quote}
    \emph{When the ledger collapses multiplicative chaos into additive 
    consistency, and recursion governs transitions, physics becomes the 
    shadow of self-consistent knowing.}
\end{quote}
Practically, this means that once scale, recursion, and compactness are fixed,
all “theories” become coordinate projections on a single ledger rather than
independent ontological frameworks. The pedagogy centers on choosing the right
projection, not multiplying models.

\subsection{All domains emerge from}
\begin{equation}
    \text{Scale} + \text{Recursion} + \text{Compactness}.
\end{equation}

% ===========================================================
\section{Summary (A $\to$ Z)}
% ===========================================================
\begin{description}[font=\bfseries, style=multiline, leftmargin=1cm]
    \item[A] Axioms define recursion and scale.
    \item[B] Bits measure physical distance.
    \item[C] Compactness unifies GR and Newton.
    \item[D] Diffusion governs entropy.
    \item[E] Entanglement becomes geometry.
    \item[F] Finite scale gap (128 bits).
    \item[G] GR = recursion of compactness.
    \item[H] Horizon = $s=0$.
    \item[I] Information = density on ledger.
    \item[J] Joints of physics = domain interfaces.
    \item[K] Knowing replaces theory.
    \item[L] Ledger is the backbone.
    \item[M] Mass is recursion weight.
    \item[N] Newtonian limit is $u\ll1$.
    \item[O] Onu operator organizes scales.
    \item[P] Page curve = triangular recursion.
    \item[Q] Quantum = small-$n$ curvature.
    \item[R] Redshift invertible.
    \item[S] Schwarzschild smooth in $s$.
    \item[T] Thermodynamics is diffusion.
    \item[U] Unitarity = ledger continuity.
    \item[V] Vacuum = $n$-flat region.
    \item[W] Work = gradient in $n$.
    \item[X] Cross-domain compactness.
    \item[Y] Yield = recursion slope.
    \item[Z] Zero-point anchored at $n=0$.
\end{description}

\newpage
\appendix
% ===========================================================
\section{The Onu Protocol: A 6-Day Syllabus}
% ===========================================================

The Onu Calculus is designed to be learned sequentially. The following protocol
installs the necessary cognitive operators in six stages.

\subsection*{Day 1: The Ledger (Reformatting Space)}
\textbf{Objective:} Replace the meter with the Bit. \\
\textbf{Focus:} Axiom A. \\
\textbf{Core Task:} Map the Observable Universe to the finite interval 
$n \in [-12.5, 115.6]$. Realize that distance is simply \emph{recursion depth}.

\subsection*{Day 2: The Move (Reformatting Math)}
\textbf{Objective:} Replace subtraction with The Onu Operator. \\
\textbf{Focus:} Axiom B. \\
\textbf{Core Task:} Internalize the \emph{Onu Difference} $a \ominus b = a/b$. 
Prove that a physical doubling is a single-bit shift ($\Delta n = -1$). 
Physics becomes the study of bit-costs.

\subsection*{Day 3: The Engine (Reformatting Gravity)}
\textbf{Objective:} Replace Force with Compactness. \\
\textbf{Focus:} Axiom C. \\
\textbf{Core Task:} Master the parameter $u = 2GM/Rc^2$. 
Identify the three zones: Newtonian ($u \ll 1$), Relativistic ($u \approx 1/3$), 
and Horizon ($u \to 1$).

\subsection*{Day 4: The Choice (Reformatting Logic)}
\textbf{Objective:} Replace Theory with Domain Priority. \\
\textbf{Focus:} The Selection Rule $D(R)$. \\
\textbf{Core Task:} Use the maximization of $u$ to deterministically select 
between Quantum and Gravitational descriptions without philosophical ambiguity. 
Derive the Planck Mass as the intersection of these domains.

\subsection*{Day 5: The Flow (Reformatting Entropy)}
\textbf{Objective:} Replace Paradox with Diffusion. \\
\textbf{Focus:} Information Flow. \\
\textbf{Core Task:} Transform the Event Horizon from a singularity into the 
origin $s=0$. Model Black Hole evaporation as simple heat diffusion on the 
Scale Ledger.

\subsection*{Day 6: The Synthesis (Knowing)}
\textbf{Objective:} Unified Knowing. \\
\textbf{Focus:} The A-Z Summary. \\
\textbf{Core Task:} Analyze a high-complexity event (e.g., GW150914) entirely 
within the Onu framework, calculating Scale Shifts, Compactness Trajectories, 
and Recursion Costs.

\end{document}
\section{Axioms of the Onu Calculus}
% ===========================================================

The Onu Calculus begins from three axioms. These define structure; 
all physics-like behaviors emerge as corollaries.

\subsection{Axiom A (Scale)}
Every measurable extent $r\in \R^+$ is mapped to an additive scale index
\begin{equation}
    n(r) = -\log_2\!\left(\frac{r}{r_0}\right), 
\end{equation}
where $r_0$ is a reference length (taken as $1\,$m unless noted).
This map is bijective and generates the \emph{scale ledger} $\sL$.

\subsection{Axiom B (Recursion)}
Any quantity $X(r)$ evolving multiplicatively over $r$ evolves additively over $n$:
\begin{equation}
    X(2^{-n}) = e^{-kn} X(1),
\end{equation}
for some scale-coupling $k$.  
This is the fundamental recursion principle.

\subsection{Axiom C (Compactness)}
All physically relevant dynamics depend on a 
\textbf{single compactness parameter}
\begin{equation}
    u = \frac{2GM}{R c^2},
\end{equation}
which governs Newton, GR, redshift, diffusion, and information flow.

Everything else is `knowing projections’.

% ===========================================================
\section{Scale Ledger: Additive Physics}
% ===========================================================

The ledger transform converts extreme multiplicative hierarchies into 
finite additive structures.

\subsection{Planck–Astro Gap}
The Planck length $\ellP = 1.616\times10^{-35}$ m maps to
\begin{equation}
    n(\ellP) \approx 115.6.
\end{equation}

A neutron-star gravitational radius $\sim 10^4$ m yields
\begin{equation}
    n(\rg) \approx -12.5.
\end{equation}

Thus the entire quantum-to-relativistic universe spans
\begin{equation}
    \Delta n = 128.1 \text{ bits}.
\end{equation}

This finiteness is the basis for unification.

% ===========================================================
\section{Compactness Interfaces: Newton → GR}
% ===========================================================

\subsection{Newtonian compactness}
\begin{equation}
    u = \frac{2GM}{R c^2}, \qquad 0 < u < 1.
\end{equation}

In Newtonian regime $u \ll 1$,
\begin{equation}
    \Phi_N = -\frac{GM}{R}, \qquad v_{\rm esc} = c \sqrt{u}.
\end{equation}

\subsection{GR corrections}
\begin{align}
    g_{\rm GR}(R) &= \frac{GM}{R^2 \sqrt{1-u}}, \\
    z_{\rm GR} &= (1-u)^{-1/2} - 1.
\end{align}

\subsection{Redshift inversion}
\begin{equation}
    u = 1 - (1+z_{\rm GR})^{-2}.
\end{equation}

This is the key link between information, geometry, and recursion.

% ===========================================================
\section{Onu Information Flow}
% ===========================================================

\subsection{Ledger diffusion}
On the scale-ledger, information density $\I(n,t)$ evolves via
\begin{equation}
    \frac{\partial \I}{\partial t} = D \frac{\partial^2 \I}{\partial n^2},
\end{equation}
a 1D diffusion equation.

Black-hole evaporation reduces to diffusion with a boundary condition at $n_s$,
the horizon scale.

\subsection{Page curve recursion}
The Page curve becomes a triangle:
\begin{equation}
    S(t) = S_{\rm max}\,\min\!\left(\frac{t}{t_{\rm Page}}, 2 - \frac{t}{t_{\rm Page}}\right).
\end{equation}

No new physics is added; the ledger removes the coordinate singularity.

% ===========================================================
\section{Unified Horizon Recursion}
% ===========================================================

Define ledger coordinate relative to the Schwarzschild radius:
\begin{equation}
    s = \ln\left(\frac{r}{r_s}\right).
\end{equation}

Then:
\begin{itemize}
    \item $s = 0$ is the horizon,
    \item $s < 0$ interior,
    \item $s > 0$ exterior,
    \item all divergent GR functions become smooth functions on $s$.
\end{itemize}

Example:
\begin{equation}
    (1 - r_s/r) = 1 - e^{-s}.
\end{equation}

The interior/exterior ``duality’’ is simply reflection across $s=0$.

% ===========================================================
\section{Generalized Informational Compactness}
% ===========================================================

Assume a domain-indexed extension:
\begin{equation}
    u_X(R,M) = \frac{2 K_X M}{R c^2}.
\end{equation}

This allows:
\begin{itemize}
    \item micro-informational domains,
    \item effective field compactness,
    \item recursive cross-domain comparison.
\end{itemize}

When $K_X = G$, one recovers standard gravitational compactness.

% ===========================================================
\section{Domain Priority Function}
% ===========================================================

Define the dominant domain at scale $(R,M)$ by
\begin{equation}
    D(R) = \arg\max_X u_X(R,M).
\end{equation}

This produces a deterministic selection between:
\begin{itemize}
    \item Newtonian logic,
    \item GR logic,
    \item quantum-informational logic.
\end{itemize}

There is no dynamical claim—only selection consistency.

% ===========================================================
\section{Unified Recursive Interpretation}
% ===========================================================

\subsection{Knowing instead of theory}
Onu Calculus does not posit entities.
It organizes experience:

\begin{quote}
    \emph{If the ledger collapses multiplicative chaos into additive consistency,  
    and recursion governs all transitions,  
    then physics is the shadow of self-consistent knowing.}
\end{quote}

\subsection{All domains emerge from}
\begin{equation}
    \text{Scale} + \text{Recursion} + \text{Compactness}.
\end{equation}

Newton, GR, QM, Thermodynamics, Information—all appear as projections of these.

% ===========================================================
\section{Summary (A → Z)}
% ===========================================================

\begin{enumerate}
    \item A: Axioms define recursion and scale.
    \item B: Bits measure physical distance.
    \item C: Compactness unifies GR and Newton.
    \item D: Diffusion governs entropy.
    \item E: Entanglement becomes geometry.
    \item F: Finite scale gap (128 bits).
    \item G: GR = recursion of compactness.
    \item H: Horizon = $s=0$.
    \item I: Information = density on ledger.
    \item J: Joints of physics = domain interfaces.
    \item K: Knowing replaces theory.
    \item L: Ledger is the backbone.
    \item M: Mass is recursion weight.
    \item N: Newtonian limit is $u\ll1$.
    \item O: Onu operator organizes scales.
    \item P: Page curve = triangular recursion.
    \item Q: Quantum = small-$n$ curvature.
    \item R: Redshift invertible.
    \item S: Schwarzschild smooth in $s$.
    \item T: Thermodynamics is diffusion.
    \item U: Unitarity = ledger continuity.
    \item V: Vacuum = $n$-flat region.
    \item W: Work = gradient in $n$.
    \item X: Cross-domain compactness.
    \item Y: Yield = recursion slope.
    \item Z: Zero-point anchored at $n=0$.
\end{enumerate}

\end{document}
