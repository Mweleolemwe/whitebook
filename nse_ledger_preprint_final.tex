
\documentclass[12pt]{article}
\usepackage{amsmath,amsthm,amssymb,mathtools}
\usepackage[T1]{fontenc}
\usepackage[margin=1in]{geometry}
\usepackage{hyperref}
\hypersetup{colorlinks=true,linkcolor=black,citecolor=black,urlcolor=black}

\newtheorem{theorem}{Theorem}
\newtheorem{lemma}{Lemma}
\newtheorem{proposition}{Proposition}
\newtheorem{corollary}{Corollary}
\theoremstyle{definition}
\newtheorem{definition}{Definition}
\theoremstyle{remark}
\newtheorem{remark}{Remark}

\DeclareMathOperator{\diverg}{div}
\DeclareMathOperator{\curl}{curl}

\title{Ledgered Local Energy and $\varepsilon$--Regularity for 3D Navier--Stokes}
\author{}
\date{\today}

\begin{document}
\maketitle

\begin{abstract}
We prove a ledgered local energy identity for 3D incompressible Navier--Stokes in the Leray--Hopf class. 
The identity augments the classical localized energy inequality by a nonnegative Radon measure $\mu$ (the \emph{ledger/defect measure}) which captures unresolved nonlinear flux, yielding an exact conservation law with a positive defect.
As an application, we derive a scale--invariant $\varepsilon$--regularity criterion on parabolic cylinders depending on the Caffarelli--Kohn--Nirenberg energy plus the scaled ledger mass. 
We complement the analysis with conservative discretizations that reproduce the identity to truncation order.
\end{abstract}

\section{Setting and notation}
Let $v:\Omega\times(0,T)\to\mathbb{R}^3$ and $p:\Omega\times(0,T)\to\mathbb{R}$ solve
\begin{equation}\label{NSE}
\partial_t v + (v\!\cdot\!\nabla)v + \nabla p = \nu\Delta v,\qquad \diverg v=0,
\end{equation}
in the Leray--Hopf sense on $\Omega=\mathbb{R}^3$ or $\mathbb{T}^3$ (periodic), with $v\in L^\infty_t L^2_x\cap L^2_t\dot H^1_x$ obeying the global energy inequality.
For a parabolic cylinder $Q_r(z_0)=B_r(x_0)\times(t_0-r^2,t_0)$ we set
\begin{equation*}
\mathcal{E}(r) := r^{-1}\!\int_{Q_r}\Big(|\nabla v|^2 + |p-p_{Q_r}|^{3/2}\Big)\,dx\,dt,
\qquad
\Lambda(r) := r^{-1}\,\mu(Q_r).
\end{equation*}

\section{Defect/ledger measure}
\begin{definition}[Ledger/defect measure]\label{def:ledger}
Let $\rho_\varepsilon$ be a standard (space--time) mollifier and set $v^\varepsilon=v\ast\rho_\varepsilon$.
Define the commutator density
\begin{equation*}
d\mu_\varepsilon := \Big[(v\otimes v)^\varepsilon - v^\varepsilon\!\otimes v^\varepsilon\Big]\!:\!\nabla v^\varepsilon\,dx\,dt.
\end{equation*}
Along a sequence $\varepsilon\downarrow0$ we have $d\mu_\varepsilon\rightharpoonup d\mu$ in $\mathcal{M}_{\mathrm{loc}}(\Omega\times(0,T))$ with $d\mu\ge0$. 
We call $\mu$ the \emph{ledger (defect) measure}. (This coincides with the Duchon--Robert defect.)
\end{definition}

\section{Ledgered local energy identity}
\begin{theorem}[Leray--Hopf ledgered local energy]\label{thm:ledger}
There exists a nonnegative Radon measure $\mu$ such that the distributional identity
\begin{equation}\label{eq:DR}
\partial_t \Big(\tfrac12|v|^2\Big) + \nabla\!\cdot\!\Big[\Big(\tfrac12|v|^2+p\Big)v\Big]
= \nu\,\Delta\!\Big(\tfrac12|v|^2\Big) - \nu|\nabla v|^2 - \mu
\end{equation}
holds on $\Omega\times(0,T)$. Equivalently, for $\phi\in C_c^\infty(\Omega\times[0,T))$, $\phi\ge0$,
\begin{align}\label{eq:ledger-weak}
&\int \tfrac12|v|^2 \phi(\cdot,t)\,dx + \nu\!\int_0^t\!\!\int |\nabla v|^2 \phi\,dx\,ds + \int_0^t\!\!\int \phi\,d\mu \nonumber\\
&= \int_0^t\!\!\int \Big(\tfrac12|v|^2+p\Big) v\!\cdot\!\nabla\phi \;+\; \tfrac12|v|^2(\partial_s\phi+\nu\Delta\phi)\,dx\,ds
 + \int \tfrac12|v|^2 \phi(\cdot,0)\,dx .
\end{align}
\end{theorem}

\section{$\varepsilon$--regularity with ledger saturation}
\begin{theorem}[$\varepsilon$--regularity]\label{thm:ereg}
There exist universal constants $\varepsilon_*,\delta>0$ such that: if for some cylinder $Q_{r_0}(z_0)$
\begin{equation}\label{eq:smallness}
\mathcal{E}(r_0) + \Lambda(r_0) \le \varepsilon_*,
\end{equation}
then $v$ is H\"older continuous on $Q_{\delta r_0}(z_0)$.
\end{theorem}

\section{Discrete mirror}
On $\mathbb{T}^d$ ($d=2,3$) with skew--symmetric convection and a divergence-free projection, an RK2 step gives
\begin{equation*}
E^{n+1}-E^n = -\nu\,\Delta t\,\|\omega^{n+\frac12}\|_2^2 - L^n + \mathcal{O}(\Delta t^3),
\qquad
E(t)+\int_0^t \nu\|\omega\|_2^2 = E(0) + \mathcal{O}(\Delta t^2).
\end{equation*}
\end{document}
