% !TEX program = lualatex
\documentclass[11pt, a4paper]{article}

% ---------- CORE PACKAGES ----------
\usepackage{fontspec} % required for lualatex (replaces inputenc/fontenc)
\usepackage[margin=1in]{geometry}
\usepackage{microtype}
\usepackage{amsmath,amssymb,amsthm,mathtools,bm}
\usepackage{booktabs}
\usepackage{graphicx}
\usepackage[hidelinks]{hyperref}
\usepackage{enumitem}
\usepackage{listings}
\usepackage{xcolor}
\usepackage{siunitx}

% --- BIBLATEX SETUP (Option A) ---
\usepackage[backend=biber,style=phys]{biblatex}
\addbibresource{references.bib} % Looks for references.bib file
% ---------------------------------

% ---------- NEW PARTITION OPERATORS (CRITICAL) ----------
% a \partition b denotes the partition (ratio) a/b on R_{>0}
\newcommand{\partition}{\mathbin{\oslash}} 
% s_1 \lpart s_2 denotes the ledger partition of two scale-coordinates
\newcommand{\lpart}{\mathbin{\vcentcolon\!\vartriangleleft}} 

% ---------- THEOREM STYLES ----------
\newtheorem{definition}{Definition}
\newtheorem{proposition}{Proposition}
\newtheorem{axiom}{Axiom}
\newtheorem{theorem}{Theorem} % Möbius Gauge Theorem, etc.
\newtheorem{remark}{Remark}

% ---------- SHORTCUTS AND NOTATION ----------
\newcommand{\dd}{\mathrm d}         % differential
\newcommand{\Rpos}{\mathbb{R}_{>0}} % positive domain
\newcommand{\E}{\mathcal{E}}        % Euler operator z\partial_z
\newcommand{\RLO}{\mathcal{R}}      % Scale Laplacian (z\partial_z)^2
\newcommand{\Var}{\operatorname{Var}}
\newcommand{\Varlog}{\Var_{\log}^{\mathrm{obs}}}
\newcommand{\Lsac}{\mathcal{L}_{\text{sacred}}}
\newcommand{\tauf}{\tau_f}
\newcommand{\OnuOp}{\mathcal{O}_{\mathrm{Onu}}} % The Breathing Operator
\newcommand{\ONU}{\mathrm{ON\acute{U}}} % The Stationary Limit
\newcommand{\nref}[2]{n_{#1}\!\left(#2\right)} % Scale-ledger coord

% ---------- COLORS & LISTINGS (for Python code) ----------
\definecolor{codegray}{rgb}{0.5,0.5,0.5}
\definecolor{codegreen}{rgb}{0,0.6,0}
\lstset{
    language=Python,
    basicstyle=\ttfamily\footnotesize,
    numbers=left,
    numberstyle=\tiny\color{codegray},
    stepnumber=1,
    numbersep=5pt,
    backgroundcolor=\color{white},
    showspaces=false,
    showstringspaces=false,
    frame=single,
    rulecolor=\color{black},
    commentstyle=\color{codegreen},
    keywordstyle=\color{blue},
    stringstyle=\color{codegray},
    breaklines=true,
    breakatwhitespace=true,
    tabsize=2
}

% ---------- TITLE AND METADATA ----------
\title{\textbf{The Onu-Ma'at Synthesis: A Unified Field Pedagogy}\\\\
\large Final Axiomatization via Partition Calculus and Coherence Refraction}
\author{Y. Mwesgoleo (Onu Collective)}
\date{Final Coalescence: \today}

% =======================================================
%               DOCUMENT START
% =======================================================
\begin{document}
\maketitle

\begin{abstract}
We present the Onu-Ma'at Synthesis, a unified field pedagogy. The framework is anchored in a strictly positive \textbf{Partition Calculus} on a log-scale M\"obius Ledger, eliminating the concept of erasure or subtraction. We define the stationary computational frame by the \textbf{$\ONU$ limit} and reinterpret gravitational curvature as \textbf{refraction} driven by gradients in coherence pressure ($P_{\mathrm{coh}}$). The dynamic core is the $\OnuOp$ (Breathing Operator), which ensures conservation and stabilizes the system across scales. The theory is fixed by three constants ($V_0, k, \Xi$) and culminates in an incontrovertible, falsifiable test for the Zayyay Floor ($\Xi$) in gravitational wave data.
\end{abstract}

\vspace{0.5cm}
\begin{center}
\textbf{The Word of the Day is: INCONTROVERTIBLE (Ppwsot)}
\end{center}
\vspace{0.5cm}

% =======================================================
\section{Pedagogical Overview: Stationary Frame and Coherence Refraction}
% =======================================================

We ascend to a stationary computational frame defined by the \emph{availability of propagating amplitude}. This **$\ONU$ limit** (the fixed point where amplitude emission and extinction are equivalent) is the closest attainable analog to a stationary frame, where metric deformation induced by local amplitude availability vanishes.

Between $\ONU$ and the observer frame, phenomena are governed by the coherence-pressure tensor, $P_{\mathrm{coh}}$. This pressure defines the operational mass density,
\begin{equation}
\rho_{\mathrm{mass}} \;=\; \nabla \cdot P_{\mathrm{coh}},
\end{equation}
and induces a refractive structure. This allows gravitational curvature to be interpreted as a **refractive effect** rather than intrinsic geometric warping.

% =======================================================
\section{The Unifying Principle: The Onu Partition Calculus}
% =======================================================

The entire framework is built on a strictly positive domain $\Rpos$. We replace arithmetic differences with relational partitions.

\begin{definition}[Onu Partition and Scale Ledger]
For $a,b\in\Rpos$, the \emph{Onu partition} is the ratio $a \partition b := a/b$. Fixing a reference $r_0$, the \emph{scale-ledger coordinate} is $\nref{r_0}{r} := -\log_2(r/r_0)$.
\end{definition}

\begin{definition}[Ledger-Partition Operator]
On the scale-ledger (the logarithmic band $\mathcal{S}$), we define the \emph{ledger-partition operator} ($\lpart$) algebraically as:
\[
s_1 \lpart s_2 \;\overset{\text{def}}{=}\; s_1 - s_2.
\]
\end{definition}

\begin{theorem}[Möbius Gauge Equivalence]
The relational content of the system is invariant under a shift of the reference scale. For any two references $r_0,r_0'\in\Rpos$ and any entries $r_1, r_2\in\Rpos$, the ledger-partition is gauge-invariant:
\[
\nref{r_0'}{r_1} \lpart \nref{r_0'}{r_2}
=
\nref{r_0}{r_1} \lpart \nref{r_0}{r_2}.
\]
\end{theorem}

\begin{figure}[h]
    \centering
    \includegraphics[width=0.9\textwidth]{figures/two_ways_to_onu.png}
    \caption{\textbf{Two Ways to Onu.} Left: Flow the M\"obius through a fixed Onu knot (adjust entries). Right: Move the Onu knot along a fixed ledger (adjust reference). The ledger-partitions remain invariant.}
\end{figure}

% =======================================================
\section{Proposition I: The Explicit Linear Law of Spectral Coherence}
% =======================================================

This law governs the stability of the microscopic emotional field in the octanodal container ($Y_3^{\pm 2}$).

\begin{proposition}[Explicit Linear Law]
The observed log-variance is linearly related to the leakage fraction $(1-C)$:
\begin{equation}
\boxed{\; \Varlog \;=\; \mathbf{V_0} + \mathbf{k}(1-C) \;}
\end{equation}
\end{proposition}

\begin{definition}[Lule’s Constant]
\textbf{Lule’s Constant ($V_0 = 0.170$)} is the irreducible log-variance ($v_{\mathrm{in}}$) of the perfectly contained state.
\end{definition}

% =======================================================
\section{Proposition II: The Galactic Law and the Zayyay Floor}
% =======================================================

\begin{proposition}[The Galactic Law via Refractive Gradient]
The circular velocity $v_c(r)$ is given by:
\begin{equation}
v_c(r) \;=\; \mathcal{V} \sqrt{\frac{\alpha r}{r + r_c}}
\end{equation}
\end{proposition}

\begin{definition}[Zayyay’s Constant and The Cosmic Floor]
\textbf{Zayyay’s Constant ($\Xi = 0.0041$)} is the minimum pressure factor threshold ($\tau_{\min}/\tau_0$).
\end{definition}

% =======================================================
\section{The Onu Breathing Operator ($\OnuOp$)}
% =======================================================

\begin{definition}[Onu Breathing Operator]
The \emph{Onu breathing operator} $\OnuOp$ enforces boundary contact between the vacuum and ultra-density limits, modulated by $\kappa(t)$.
\[
\OnuOp[f](s,t) \;\equiv\; \kappa(t)\,\frac{1}{2}\Big( \lim_{S\to+\infty} \mathcal{T}_{S}f(s) + \lim_{S\to-\infty}\mathcal{T}_{S}f(s)\Big),
\]
\end{definition}
The acceleration field is corrected by:
\begin{equation}
\mathbf{a}(r,t) \;=\; -\mathcal{V}^2 \nabla \ln \tauf(r) \;-\; \lambda\,\nabla_s\!\big[\OnuOp(s,t)\big]\!.
\end{equation}

% =======================================================
\section{Incontrovertible Test: The Zayyay Floor on Black Hole Spin}
% =======================================================

\textbf{Hypothesis (Incontrovertible):} The effective spin parameter $\chi_{\text{eff}}$ of gravitational wave events must exhibit a physical floor corresponding to the Zayyay Constant:
$$
|\chi_{\text{eff}}^{\text{floor}}| \; \approx \; \Xi \;=\; 0.0041
$$

\begin{lstlisting}[caption={Python script to check Zayyay floor on $|\chi_{\text{eff}}|$}, label={lst:zayyay_test}]
import h5py
import numpy as np
import matplotlib.pyplot as plt

ZAYYAY_CONSTANT = 0.0041 

def check_zayyay_floor(file_path):
    # Load HDF5 and extract chi_eff from posterior_samples
    # ... (code for data extraction from NRSur/SEOBNR .h5 files) ...
    abs_chi_eff = np.abs(chi_eff_samples)
    
    # Visualization and quantitative check
    plt.hist(abs_chi_eff, bins=50, density=True)
    plt.axvline(ZAYYAY_CONSTANT, color='red', linestyle='--')
    plt.xlim(0, 0.05) 
    plt.show()
\end{lstlisting}

% =======================================================
\section{Registry of Constants}
% =======================================================

\begin{center}
\begin{tabular}{llll}
\toprule
\textbf{Symbol} & \textbf{Name} & \textbf{Value} & \textbf{Domain} \\
\midrule
$V_0$ & Lule’s Constant & $0.170$ & Microscopic Coherence \\
$k$ & Leakage Coefficient & $0.10$ & Mesoscopic Transport \\
$\Xi$ & Zayyay’s Constant & $0.0041$ & Macroscopic Refractive Floor \\
\bottomrule
\end{tabular}
\end{center}

\section*{Acknowledgements (Topological Boundary Condition)}
We acknowledge the Onu Collective, the unseen hand of LuDu, and the data from NRSur/SEOBNR. The work is topologically anchored by the presence of Mount Shasta, our reference for stationary amplitude.
\begin{figure}[h]
    \centering
    \includegraphics[width=0.5\textwidth]{figures/mount_shasta.png}
    \caption{Mount Shasta: The topological boundary condition for the Onu Collective's locus of operation.}
\end{figure}

\printbibliography
\end{document}
