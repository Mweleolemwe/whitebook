% !TEX program = lualatex
\documentclass[11pt]{article}

% ---------- PACKAGES ----------
\usepackage[margin=1in]{geometry}
\usepackage{microtype}
\usepackage{amsmath,amssymb,amsthm,mathtools}
\usepackage{bm}
\usepackage{graphicx}
\usepackage[hidelinks]{hyperref}
\usepackage{enumitem}
\usepackage{listings}
\usepackage{xcolor}
\usepackage{libertinus}
\usepackage[capitalize,nameinlink]{cleveref}
\usepackage[backend=biber,style=phys]{biblatex}
\addbibresource{references.bib}

% ---------- TYPOGRAPHY ----------
\linespread{1.05}
\numberwithin{equation}{section} % keeps equations organized by section

% ---------- COLORS & LISTINGS ----------
\definecolor{codegray}{rgb}{0.5,0.5,0.5}
\definecolor{codegreen}{rgb}{0,0.6,0}
\lstset{
    language=Python,
    basicstyle=\ttfamily\footnotesize,
    numbers=left,
    numberstyle=\tiny\color{codegray},
    stepnumber=1,
    numbersep=5pt,
    backgroundcolor=\color{white},
    showspaces=false,
    showstringspaces=false,
    showtabs=false,
    frame=single,
    rulecolor=\color{black},
    commentstyle=\color{codegreen},
    keywordstyle=\color{blue},
    stringstyle=\color{codegray},
    breaklines=true,
    breakatwhitespace=true,
    tabsize=3
}

% ---------- MATH OPERATORS ----------
\DeclareMathOperator{\Var}{Var}
\DeclareMathOperator{\logvar}{Var_{\log}}

% ---------- SHORTCUTS ----------
\newcommand{\dd}{\mathrm{d}}
\newcommand{\Rpos}{\mathbb{R}_{>0}}
\newcommand{\Onu}{\mathcal{O}} % Onu gauge symbol if you need it explicitly
\newcommand{\sL}{\mathcal{S}}
\newcommand{\nref}[2]{n_{#1}\!\left(#2\right)}
\newcommand{\tauf}{\bm{\tau}}
\newcommand{\ellp}{\ell_p}
\newcommand{\ellz}{\ell_{(0)}}
\newcommand{\vc}{v_c}
\newcommand{\Quasmos}{\textbf{Quasmos}}
\newcommand{\TTC}{\textbf{TTC}}
\newcommand{\Maat}{\textbf{Ma'at}}

% ---------- THEOREM STYLES ----------
\newtheorem{definition}{Definition}[section]
\newtheorem{theorem}{Theorem}[section]
\newtheorem{remark}{Remark}[section]
\newtheorem{proposition}{Proposition}[section]

% ---------- TITLE ----------
\title{\textbf{The Quasmos: A Unified Field Theory of Time-Topology,\\
Onu Recursive Gauge, and Atomic Stability}}
\author{Yay / Onu Collective \\[4pt]
\small Independent Research Group, 2025}
\date{\today}

% ===========================================================
\begin{document}
\maketitle

\begin{abstract}
We propose the \Quasmos\ framework, describing the universe as a continuous, viscous log-domain ledger of time-density ($\tauf$). The theory aims to unify gravitational, fluid, and atomic dynamics via a logarithmic recursion gauge (Onu). Time-Topology Cosmology (TTC) replaces dark matter with gradients in $\tauf$, while the Ma’at atomic model arises as a stable variance minimum in the same field. The sliding Planck threshold $\ellp(S) = \ellz\,\delta^S$ is introduced to ensure scale continuity from cosmic to quantum domains.
\end{abstract}

% ===========================================================
\section{Introduction}
The standard cosmological and quantum frameworks introduce dark matter and discrete quantum postulates to explain galactic rotation, structure formation, and atomic stability. In this work we explore an alternative approach in which a single time-density field $\tauf$ and its logarithmic structure underlie both gravitational and atomic phenomena.

We outline the Onu logarithmic gauge, apply it at galactic scales via Time-Topology Cosmology (\TTC), and introduce the Ma’at atomic configuration as a variance-minimizing structure in the same field. A scale-dependent Planck threshold is then used to connect these regimes continuously.

% ===========================================================
\section{The Onu Triad: Log-Linear Recursion}\label{sec:onu-triad}

\begin{definition}[Time-density field]
Let $M$ be a spacetime slice (e.g.\ a constant-cosmic-time hypersurface). 
A \emph{time-density field} is a positive scalar field
\[
    \tauf : M \to \Rpos,
\]
interpreted as the density of proper-time accumulation per unit volume.
We will work in regimes where $\ln \tauf$ is smooth and its spatial gradients are well-defined.
\end{definition}

The Onu gauge posits that complex dynamics, traditionally described by curvature or non-linear terms, simplify to linear relations in a logarithmic domain
\begin{equation}
    s = \ln\!\left(\frac{r}{r_s}\right),
\end{equation}
so that for an observable $X$,
\begin{equation}
    \ln X(s) = \ln X_0 + \kappa s.
\end{equation}
This is used to unify:
\begin{enumerate}[label=(\roman*)]
    \item \textbf{Gravity:} $\ln u(s) \propto -s$ (horizon gauge),
    \item \textbf{Fluid flow:} $\ln v(s) \propto 2\beta s$ (Hawking--Squeeze),
    \item \textbf{Arithmetic:} $\Rpos$ as a twisted M\"obius ledger on $\nref{r_0}{r}$.
\end{enumerate}

\begin{definition}[Onu scale ledger]
For a chosen reference radius $r_0 \in \Rpos$, the Onu scale ledger is
\[
    \nref{r_0}{r} := \ln\!\left(\frac{r}{r_0}\right),
\]
defined up to an additive constant. Any other choice related by 
$\nref{r_0'}{r} = \nref{r_0}{r} + C(r_0',r_0)$ is called a M\"obius gauge transform.
\end{definition}

% ===========================================================
\section{M\"obius Gauge Equivalence}\label{sec:mobius}

\begin{theorem}[Onu M\"obius Gauge Equivalence]
For any $r_0, r_0' \in \Rpos$,
\begin{equation}
    \nref{r_0'}{r} = \nref{r_0}{r} + C(r_0', r_0),
\end{equation}
where $C$ is constant in $r$. Ledger differences are invariant: if $r_1, r_2 > 0$,
\begin{equation}
    \nref{r_0'}{r_2} - \nref{r_0'}{r_1}
    =
    \nref{r_0}{r_2} - \nref{r_0}{r_1}.
\end{equation}
\end{theorem}

\begin{proof}[Sketch of proof]
Using the definition 
$\nref{r_0}{r} = \ln(r/r_0)$, we have
\[
    \nref{r_0'}{r} 
    = \ln\!\left(\frac{r}{r_0'}\right)
    = \ln\!\left(\frac{r}{r_0}\right) + \ln\!\left(\frac{r_0}{r_0'}\right)
    = \nref{r_0}{r} + C(r_0',r_0),
\]
with $C(r_0',r_0) := \ln(r_0/r_0')$ independent of $r$. Subtracting two radii $r_1,r_2$ cancels $C$, proving gauge invariance of ledger differences.
\end{proof}

% ===========================================================
\section{Macro-Scale: Time-Topology Cosmology (\TTC)}\label{sec:ttc}
In \TTC, gravitational acceleration arises from the log-gradient of $\tauf$:
\begin{equation}
    \bm{a}(r) = -c^2 \nabla \ln \tauf(r), 
    \qquad 
    \vc(r) = \sqrt{|\bm{a}(r)|\,r}.
\end{equation}

As a toy model, consider a spherically symmetric profile
\begin{equation}
    \tauf(r) = \tauf_0 \left(1 + \frac{r}{r_c}\right)^{-\alpha},
    \qquad \alpha > 0,
\end{equation}
with core radius $r_c$. Then
\begin{equation}
    \frac{\dd}{\dd r} \ln \tauf(r) = -\frac{\alpha}{r + r_c},
\end{equation}
so that in the radial direction
\begin{equation}
    a(r) = -c^2 \frac{\alpha}{r + r_c},
    \qquad
    v_c(r) = c \sqrt{\frac{\alpha r}{r + r_c}}.
\end{equation}
For $r \gg r_c$, the rotation curve approaches a constant
$v_c(r) \to c\sqrt{\alpha}$, reproducing flat rotation curves without postulating dark matter.

\begin{figure}[h]
    \centering
    \includegraphics[width=0.7\textwidth]{rotation_curve.pdf}
    \caption{Example galactic rotation curve from the Onu-gradient law for a simple $\tauf(r)$ profile.}
    \label{fig:rotation-curve}
\end{figure}

% ===========================================================
\section{Micro-Scale: Ma’at Atomic Stability}\label{sec:maat}
Matter corresponds to a local minimum of the variance of $\tauf$:
\begin{equation}
    \boxed{\logvar(C) = V_0 + k(1-C)}, 
    \qquad
    V_0 \approx 0.170,\quad k \approx 0.10.
\end{equation}
Here $C \in [0,1]$ is a coherence parameter encoding how tightly the local time-density trajectories align with the Ma’at 8-horn topology: $C=1$ denotes perfect alignment (minimal variance), while $C=0$ corresponds to maximal dispersion in the local time-field. The linear dependence of $\logvar$ on $(1-C)$ mirrors the log-linear Onu recursion used at galactic scales.

This defines the Ma’at atom: a stable 8-horn topology representing minimal entropy in the time-field.

% ===========================================================
\section{Unification: The Sliding Planck Threshold}\label{sec:planck}
We introduce a scale-dependent Planck threshold
\begin{equation}
    \ellp(S) = \ellz \cdot \delta^S, 
    \qquad 0<\delta<1,
\end{equation}
to enforce a continuous scale bridge from cosmic gradients to atomic stability. Taking logs,
\begin{equation}
    \ln \ellp(S) = \ln \ellz + S \ln \delta,
\end{equation}
so that $S$ plays the role of an Onu ledger step: changing $S$ by one unit shifts $\ln \ell_p$ by a constant. Traversing from galactic to atomic scales is thus a finite walk on the same log ledger, rather than a discontinuous jump between unrelated regimes.

\begin{figure}[h]
    \centering
    \includegraphics[width=0.75\textwidth]{tau_field_continuity.pdf}
    \caption{Schematic of $\tauf$ continuity from cosmic to atomic domains.}
    \label{fig:tau-continuity}
\end{figure}

% ===========================================================
\section{Discussion and Outlook}\label{sec:discussion}
We have sketched a unified view in which a single time-density field $\tauf$ underlies both galactic dynamics (via \TTC) and atomic stability (via the Ma’at configuration), with all regimes governed by the same log-linear Onu recursion on an underlying ledger.

Several open questions remain:
\begin{itemize}
    \item Deriving \TTC\ rotation curves that fit specific galaxies quantitatively and confronting standard dark-matter halo fits.
    \item Connecting the Ma’at variance minimum to known atomic spectra or coupling constants.
    \item Embedding the Onu gauge in or alongside general relativity and quantum field theory, clarifying when the log-domain description is exact and when it is an effective approximation.
\end{itemize}
Future work will focus on constructing explicit solutions for $\tauf$ at multiple scales, refining the Ma’at coherence parameter $C$ from data, and confronting the resulting predictions with observational and experimental constraints.

% ===========================================================
\appendix

\section{Appendix A: TTC Simulation Snippet (\texttt{ttc\_sim.py})}
\lstinputlisting[caption={TTC Onu-gradient law implementation}, label={lst:ttc}]{ttc_sim.py}

% ===========================================================
\printbibliography

\end{document}
