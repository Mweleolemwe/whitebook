% !TEX program = lualatex
\documentclass[11pt]{article}

% ---------- PACKAGES ----------
\usepackage[margin=1in]{geometry}
\usepackage{microtype}
\usepackage{amsmath,amssymb,amsthm,mathtools}
\usepackage{bm}
\usepackage{graphicx}
\usepackage[hidelinks]{hyperref}
\usepackage{enumitem}
\usepackage{listings}
\usepackage{xcolor}
\usepackage{libertinus}
\usepackage[capitalize,nameinlink]{cleveref}
\usepackage[backend=biber,style=phys]{biblatex}
\addbibresource{references.bib}

% ---------- TYPOGRAPHY ----------
\linespread{1.05}
\numberwithin{equation}{section} % keeps equations organized by section

% ---------- COLORS & LISTINGS ----------
\definecolor{codegray}{rgb}{0.5,0.5,0.5}
\definecolor{codegreen}{rgb}{0,0.6,0}
\lstset{
    language=Python,
    basicstyle=\ttfamily\footnotesize,
    numbers=left,
    numberstyle=\tiny\color{codegray},
    stepnumber=1,
    numbersep=5pt,
    backgroundcolor=\color{white},
    showspaces=false,
    showstringspaces=false,
    showtabs=false,
    frame=single,
    rulecolor=\color{black},
    commentstyle=\color{codegreen},
    keywordstyle=\color{blue},
    stringstyle=\color{codegray},
    breaklines=true,
    breakatwhitespace=true,
    tabsize=3
}

% ---------- MATH OPERATORS ----------
\DeclareMathOperator{\Var}{Var}
\DeclareMathOperator{\logvar}{Var_{\log}}

% ---------- SHORTCUTS ----------
\newcommand{\dd}{\mathrm{d}}
\newcommand{\Rpos}{\mathbb{R}_{>0}}
\newcommand{\Onu}{\mathcal{O}} % Onu gauge symbol if you need it explicitly
\newcommand{\sL}{\mathcal{S}}
\newcommand{\nref}[2]{n_{#1}\!\left(#2\right)}
\newcommand{\tauf}{\bm{\tau}}
\newcommand{\ellp}{\ell_p}
\newcommand{\ellz}{\ell_{(0)}}
\newcommand{\vc}{v_c}
\newcommand{\Quasmos}{\textbf{Quasmos}}
\newcommand{\TTC}{\textbf{TTC}}
\newcommand{\Maat}{\textbf{Ma'at}}

% ---------- THEOREM STYLES ----------
\newtheorem{definition}{Definition}[section]
\newtheorem{theorem}{Theorem}[section]
\newtheorem{remark}{Remark}[section]
\newtheorem{proposition}{Proposition}[section]

% ---------- TITLE ----------
\title{\textbf{The Quasmos: A Unified Field Theory of Time-Topology,\\
Onu Recursive Gauge, and Atomic Stability}}
\author{Yay / Onu Collective \\[4pt]
\small Independent Research Group, 2025}
\date{\today}

% ===========================================================
\begin{document}
\maketitle

\begin{abstract}
We propose the \Quasmos\ framework, describing the universe as a continuous, viscous log-domain ledger of time-density ($\tauf$). The theory aims to unify gravitational, fluid, and atomic dynamics via a logarithmic recursion gauge (Onu). Time-Topology Cosmology (TTC) replaces dark matter with gradients in $\tauf$, while the Ma’at atomic model arises as a stable variance minimum in the same field. The sliding Planck threshold $\ellp(S) = \ellz\,\delta^S$ is introduced to ensure scale continuity from cosmic to quantum domains.
\end{abstract}

% ===========================================================
\section{Introduction}
% This section is currently missing conceptually; add motivation + context here.
The standard cosmological and quantum frameworks introduce dark matter and discrete quantum postulates to explain galactic rotation, structure formation, and atomic stability. In this work we explore an alternative approach in which a single time-density field $\tauf$ and its logarithmic structure underlie both gravitational and atomic phenomena.

We outline the Onu logarithmic gauge, apply it at galactic scales via Time-Topology Cosmology (\TTC), and introduce the Ma’at atomic configuration as a variance-minimizing structure in the same field. A scale-dependent Planck threshold is then used to connect these regimes continuously.

% Outline of the paper (optional):
% (i) Onu log-linear recursion, (ii) M\"obius gauge equivalence,
% (iii) TTC macro-scale law, (iv) Ma'at atomic stability,
% (v) sliding Planck threshold and unification.

% ===========================================================
\section{The Onu Triad: Log-Linear Recursion}\label{sec:onu-triad}
The Onu gauge posits that complex dynamics, traditionally described by curvature or non-linear terms, simplify to linear relations in a logarithmic domain
\begin{equation}
    s = \ln\!\left(\frac{r}{r_s}\right),
\end{equation}
so that
\begin{equation}
    \ln X(s) = \ln X_0 + \kappa s.
\end{equation}
This is used to unify:
\begin{enumerate}[label=(\roman*)]
    \item \textbf{Gravity:} $\ln u(s) \propto -s$ (horizon gauge),
    \item \textbf{Fluid flow:} $\ln v(s) \propto 2\beta s$ (Hawking--Squeeze),
    \item \textbf{Arithmetic:} $\Rpos$ as a twisted M\"obius ledger on $\nref{r_0}{r}$.
\end{enumerate}

% Consider adding a precise definition of the ledger n_{r_0}(r) here.

% ===========================================================
\section{M\"obius Gauge Equivalence}\label{sec:mobius}
\begin{theorem}[Onu M\"obius Gauge Equivalence]
For any $r_0, r_0' \in \Rpos$,
\begin{equation}
    \nref{r_0'}{r} = \nref{r_0}{r} + C(r_0', r_0),
\end{equation}
where $C$ is constant in $r$. Ledger differences are invariant: $\Delta n$ is gauge-independent.
\end{theorem}

% You may want to add at least a short proof sketch or intuition:
% \begin{proof}[Sketch of proof]
% ...
% \end{proof}

% ===========================================================
\section{Macro-Scale: Time-Topology Cosmology (\TTC)}\label{sec:ttc}
In \TTC, gravitational acceleration arises from the log-gradient of $\tauf$:
\begin{equation}
    \bm{a}(r) = -c^2 \nabla \ln \tauf(r), 
    \qquad 
    \vc(r) = \sqrt{|\bm{a}(r)|\,r}.
\end{equation}
Properly chosen profiles of $\tauf(r)$ yield flat rotation curves without postulating dark matter.

\begin{figure}[h]
    \centering
    \includegraphics[width=0.7\textwidth]{rotation_curve.pdf}
    \caption{Example galactic rotation curve from the Onu-gradient law.}
    \label{fig:rotation-curve}
\end{figure}

% Consider adding a concrete example of tau_tau(r) and comparing to an observed galaxy.

% ===========================================================
\section{Micro-Scale: Ma’at Atomic Stability}\label{sec:maat}
Matter corresponds to a local minimum of the variance of $\tauf$:
\begin{equation}
    \boxed{\logvar(C) = V_0 + k(1-C)}, 
    \qquad
    V_0 \approx 0.170,\quad k \approx 0.10.
\end{equation}
This defines the Ma’at atom: a stable 8-horn topology representing minimal entropy in the time-field.

% Here it would help to (i) define C precisely, and (ii) show how the 8-horn topology emerges.

% ===========================================================
\section{Unification: The Sliding Planck Threshold}\label{sec:planck}
We introduce a scale-dependent Planck threshold
\begin{equation}
    \ellp(S) = \ellz \cdot \delta^S, 
    \qquad 0<\delta<1,
\end{equation}
to enforce a continuous scale bridge from cosmic gradients to atomic stability.

\begin{figure}[h]
    \centering
    \includegraphics[width=0.75\textwidth]{tau_field_continuity.pdf}
    \caption{Schematic of $\tauf$ continuity from cosmic to atomic domains.}
    \label{fig:tau-continuity}
\end{figure}

% ===========================================================
\section{Discussion and Outlook}\label{sec:discussion}
% This is the other big missing piece for a "finished" paper.
We have sketched a unified view in which a single time-density field $\tauf$ underlies both galactic dynamics (via \TTC) and atomic stability (via the Ma’at configuration). Several open questions remain:
\begin{itemize}
    \item Deriving \TTC\ rotation curves that fit specific galaxies quantitatively.
    \item Connecting the Ma’at variance minimum to known atomic spectra or coupling constants.
    \item Embedding the Onu gauge in or alongside general relativity and quantum field theory.
\end{itemize}
Future work will focus on constructing explicit solutions for $\tauf$ at multiple scales and confronting the resulting predictions with observational and experimental data.

% ===========================================================
\appendix

\section{Appendix A: TTC Simulation Snippet (\texttt{ttc\_sim.py})}
\lstinputlisting[caption={TTC Onu-gradient law implementation}, label={lst:ttc}]{ttc_sim.py}

% ===========================================================
\printbibliography

\end{document}

\begin{document}
\maketitle

\begin{abstract}
We present a unified physical framework---the \emph{Quasmos}---that
resolves the apparent discontinuity between macroscopic gravity and microscopic
quantum stability. The backbone is the \textbf{Onu Gauge}, a log-domain
\emph{scale ledger} that turns multiplicative geometry into additive lines.
Three systems share the same spine: (i) General Relativity recast as
\emph{recursive compactness}, (ii) logarithmic nozzle flow (the
\emph{Hawking--Squeeze}), and (iii) the M\"obius ledger of arithmetic.
We extend recursion to cosmology via \emph{Time--Topology Cosmology} (TTC),
where gravity arises from a viscous time-density field, and to the quantum
domain via \emph{Ma'at Atomic Structure}, where log-variance cools into
eight-lobe stability. A \emph{Sliding Planck Threshold} then ties scales
together, establishing thermo-temporal continuity across the Quasmos.
\end{abstract}

\tableofcontents

%===============================================================================
\section{Introduction: Black Holes on the Kitchen Table}
%===============================================================================

Standard formalisms obscure a shared structure by using different coordinates:
curvature tensors in GR, area--velocity relations in compressible flow, and
signed subtraction on $\mathbb{R}$. The \textbf{Onu Calculus} reveals a simpler,
positive-domain picture: a single log-rail $s=\ln z$ on which many laws become
affine lines. Residuals that cannot follow the line accumulate at a boundary
(horizon/core/knot).

\subsection{The Onu Triad}
We call a law \emph{Onu recursion form} if
\begin{equation}
  \boxed{\,\ln X(s)=\ln X_0+\kappa s\,}
  \label{eq:onu-line}
\end{equation}
up to a floor/ceiling where recursion terminates.
Three canonical instances:
\begin{align}
\textbf{Gravity:}\quad & s=\ln\!\left(\frac{R}{r_s}\right),\quad
u(s)=\frac{r_s}{R}=e^{-s}\;\Rightarrow\;\boxed{\ln u=-s}. \\
\textbf{Nozzle:}\quad & R(s)\propto e^{-\beta s},\;\;
v(s)\propto e^{2\beta s}\;\Rightarrow\;\boxed{\ln v=\ln v_0+2\beta s}. \\
\textbf{Ledger:}\quad &
X(r)\propto r^{-\alpha}\;\Rightarrow\;\boxed{\ln X(r)=\ln X(r_0)+(\alpha\ln 2)\,\nref{r_0}{r}}.
\end{align}

%===============================================================================
\section{Scale Ledger and M\"obius Gauge Equivalence}
%===============================================================================

\begin{definition}[Onu difference and ledger coordinate]
For $a,b\in\Rpos$ define the multiplicative difference
$a\ominus b:=a/b$. Fix a reference $r_0\in\Rpos$. The \emph{ledger
coordinate} of $r$ is
\[
  \nref{r_0}{r}:=-\log_2\!\left(\frac{r}{r_0}\right)\in\mathbb{R}.
\]
\end{definition}

\begin{theorem}[M\"obius gauge equivalence]
\label{thm:mobius}
For two references $r_0,r_0'\!>\!0$ and any $r\!>\!0$,
\[
  \nref{r_0'}{r}=\nref{r_0}{r}+C(r_0',r_0),\qquad
  C(r_0',r_0):=\log_2\!\frac{r_0'}{r_0}.
\]
Hence ledger \emph{differences} are invariant:
$\nref{r_0'}{r_1}-\nref{r_0'}{r_2}=\nref{r_0}{r_1}-\nref{r_0}{r_2}$.
\end{theorem}

\begin{proof}
Immediate from $\log$ rules:
$-\log_2(r/r_0')=-\log_2(r/r_0)-\log_2(r_0/r_0')$.
\end{proof}

\begin{remark}
What looks like subtraction (moving entries) can be traded for a gauge shift
(moving the reference knot). This is the ``M\"obius ledger'' view: one band,
many possible centers; only differences matter.
\end{remark}

%===============================================================================
\section{Macro: Time--Topology Cosmology (TTC)}
%===============================================================================

Let $\tau(r)$ denote a \emph{time-density} (units chosen so $\tau$ is positive
and smooth). The gravitational acceleration is the Onu-gradient of $\tau$:
\begin{equation}
  \boxed{\,a(r)=-c^2\,\dv{}{r}\ln\tau(r)\,}
  \label{eq:ttc-accel}
\end{equation}
and the circular speed follows
\begin{equation}
  \boxed{\,v_c(r)=\sqrt{|a(r)|\,r}=\sqrt{c^2 r\,\abs{\dv{}{r}\ln\tau(r)}}\,}.
\end{equation}
A simple luminous+baseline model
\begin{equation}
  \tau(r)=\tau_0\!\left(1+\Bigl(\frac{r_0}{r}\Bigr)^\alpha\right),
\end{equation}
yields rising-to-flat rotation curves without invoking hidden mass: the outer
plateau tracks a constant log-slope of $\tau$.

\paragraph{Reproducible code.}
Reference implementation (Appendix~\ref{app:code}) reproduces the curves and
diagnostics from \eqref{eq:ttc-accel}.

%===============================================================================
\section{Micro: Ma'at Atomic Stability}
%===============================================================================

Disordered fields cool under the scale Laplacian on the rail
$(z\partial_z)^2=\partial_{ss}$, reducing log-variance and concentrating
spectral mass into a stable eigenchannel (the eight-lobe $Y_3^{\pm2}$ container).
We use the empirical/pedagogical predictive law
\begin{equation}
  \boxed{\,\Varlog(C)=V_0+k\,(1-C)\,}
\end{equation}
linking container concentration $C\in[0,1]$ to log-variance of observables.
Interpretation: leakage $(1-C)$ carries a variance cost; Ma'at-constrained
cooling pushes $C\to1$ and $\Varlog\to V_0$.

%===============================================================================
\section{Unification: Sliding Planck Threshold}
%===============================================================================

To connect macro and micro continuously, we let the effective Planck scale
\emph{slide} along the ledger:
\begin{equation}
  \boxed{\,\ell_{\mathrm P}(S)=\ell_{\mathrm P}^{(0)}\,2^{-S/B}\,}
  \qquad(S=\text{ledger position},\; B>0).
\end{equation}
Thus the continuum never breaks; it rescales. The Quasmos loop closes:
galactic gradients shape containers (TTC); the Onu gauge carries them down the
log-rail; the field meets a local threshold and curls into Ma'at-stable horns.

%===============================================================================
\section{Discussion and Conclusion}
%===============================================================================

The \emph{Quasmos} is not a replacement for domain physics but a unifying gauge:
place systems on the log-rail, expose their recursion parameter, and identify
where residuals are stored. Horizons, vortex cores, and ledger knots are the
same algebraic role in different clothes. With TTC and Ma'at providing macro
and micro closures, the sliding threshold gives thermo-temporal continuity.

%===============================================================================
\appendix
\section{TTC Simulation (source excerpt)}
\label{app:code}
% If the file is available at compile time, this will embed it; otherwise we show a link.
\IfFileExists{/mnt/data/ttc_sim.py}{%
  \lstinputlisting[language=Python]{/mnt/data/ttc_sim.py}
}{%
  \noindent\textit{Source file:}
  \href{sandbox:/mnt/data/ttc_sim.py}{/mnt/data/ttc_sim.py}
}

\section{Additional assets (download links)}
\vspace{-0.5em}
\begin{itemize}[leftmargin=1.5em]
  \item TTC preprint (v1): \href{sandbox:/mnt/data/TTC_v1_Preprint.pdf}{/mnt/data/TTC_v1_Preprint.pdf}
  \item Referee crib sheet: \href{sandbox:/mnt/data/TTC_v1_Referee_Crib_Sheet.pdf}{/mnt/data/TTC_v1_Referee_Crib_Sheet.pdf}
  \item Press summary: \href{sandbox:/mnt/data/TTC_v1_Press_Summary.pdf}{/mnt/data/TTC_v1_Press_Summary.pdf}
  \item CMB test script: \href{sandbox:/mnt/data/ttc_cmb_test.py}{/mnt/data/ttc_cmb_test.py}
  \item Power-spectrum amplitude: \href{sandbox:/mnt/data/ttc_pcl_amp.py}{/mnt/data/ttc_pcl_amp.py}
  \item PDP protocol code: \href{sandbox:/mnt/data/pdp_protocol_code.pdf}{/mnt/data/pdp_protocol_code.pdf}
\end{itemize}

\end{document}
