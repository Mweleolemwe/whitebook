% !TEX program = lualatex
\documentclass[11pt]{article}

% ── Packages ───────────────────────────────────────────────────────────────────
\usepackage[margin=1in]{geometry}
\usepackage{microtype}
\usepackage{amsmath,amssymb,mathtools,bm}
\usepackage{amsthm}
\usepackage{physics}
\usepackage{enumitem}
\usepackage{booktabs}
\usepackage{xcolor}
\usepackage{graphicx}
\usepackage[hidelinks]{hyperref}
\usepackage{cleveref}
\usepackage{listings}

% ── Setup ──────────────────────────────────────────────────────────────────────
\lstset{
  basicstyle=\ttfamily\small,
  frame=single,
  columns=fullflexible,
  showstringspaces=false,
  keywordstyle=\color{blue!70!black},
  commentstyle=\color{green!40!black},
  stringstyle=\color{orange!60!black},
  numbers=left,
  numbersep=6pt,
  xleftmargin=1.2em
}

\newtheorem{theorem}{Theorem}
\newtheorem{remark}{Remark}
\newtheorem{definition}{Definition}
\newtheorem{proposition}{Proposition}

% ── Notation ───────────────────────────────────────────────────────────────────
\newcommand{\dd}{\mathrm{d}}
\newcommand{\Rpos}{\mathbb{R}_{>0}}
\newcommand{\onu}{\mathcal{O}}
\newcommand{\cL}{\mathcal{L}}
\newcommand{\nref}[2]{n_{#1}\!\left(#2\right)}         % ledger coordinate
\newcommand{\Varlog}{\operatorname{Var}_{\log}}       % log-variance

% ── Title block ────────────────────────────────────────────────────────────────
\title{\textbf{The Quasmos: A Unified Field Theory of Time--Topology,\\
Onu Recursive Gauge, and Atomic Stability}}
\author{Yay / Onu Collective}
\date{December 4, 2025}

\begin{document}
\maketitle

\begin{abstract}
We present a unified physical framework---the \emph{Quasmos}---that
resolves the apparent discontinuity between macroscopic gravity and microscopic
quantum stability. The backbone is the \textbf{Onu Gauge}, a log-domain
\emph{scale ledger} that turns multiplicative geometry into additive lines.
Three systems share the same spine: (i) General Relativity recast as
\emph{recursive compactness}, (ii) logarithmic nozzle flow (the
\emph{Hawking--Squeeze}), and (iii) the M\"obius ledger of arithmetic.
We extend recursion to cosmology via \emph{Time--Topology Cosmology} (TTC),
where gravity arises from a viscous time-density field, and to the quantum
domain via \emph{Ma'at Atomic Structure}, where log-variance cools into
eight-lobe stability. A \emph{Sliding Planck Threshold} then ties scales
together, establishing thermo-temporal continuity across the Quasmos.
\end{abstract}

\tableofcontents

%===============================================================================
\section{Introduction: Black Holes on the Kitchen Table}
%===============================================================================

Standard formalisms obscure a shared structure by using different coordinates:
curvature tensors in GR, area--velocity relations in compressible flow, and
signed subtraction on $\mathbb{R}$. The \textbf{Onu Calculus} reveals a simpler,
positive-domain picture: a single log-rail $s=\ln z$ on which many laws become
affine lines. Residuals that cannot follow the line accumulate at a boundary
(horizon/core/knot).

\subsection{The Onu Triad}
We call a law \emph{Onu recursion form} if
\begin{equation}
  \boxed{\,\ln X(s)=\ln X_0+\kappa s\,}
  \label{eq:onu-line}
\end{equation}
up to a floor/ceiling where recursion terminates.
Three canonical instances:
\begin{align}
\textbf{Gravity:}\quad & s=\ln\!\left(\frac{R}{r_s}\right),\quad
u(s)=\frac{r_s}{R}=e^{-s}\;\Rightarrow\;\boxed{\ln u=-s}. \\
\textbf{Nozzle:}\quad & R(s)\propto e^{-\beta s},\;\;
v(s)\propto e^{2\beta s}\;\Rightarrow\;\boxed{\ln v=\ln v_0+2\beta s}. \\
\textbf{Ledger:}\quad &
X(r)\propto r^{-\alpha}\;\Rightarrow\;\boxed{\ln X(r)=\ln X(r_0)+(\alpha\ln 2)\,\nref{r_0}{r}}.
\end{align}

%===============================================================================
\section{Scale Ledger and M\"obius Gauge Equivalence}
%===============================================================================

\begin{definition}[Onu difference and ledger coordinate]
For $a,b\in\Rpos$ define the multiplicative difference
$a\ominus b:=a/b$. Fix a reference $r_0\in\Rpos$. The \emph{ledger
coordinate} of $r$ is
\[
  \nref{r_0}{r}:=-\log_2\!\left(\frac{r}{r_0}\right)\in\mathbb{R}.
\]
\end{definition}

\begin{theorem}[M\"obius gauge equivalence]
\label{thm:mobius}
For two references $r_0,r_0'\!>\!0$ and any $r\!>\!0$,
\[
  \nref{r_0'}{r}=\nref{r_0}{r}+C(r_0',r_0),\qquad
  C(r_0',r_0):=\log_2\!\frac{r_0'}{r_0}.
\]
Hence ledger \emph{differences} are invariant:
$\nref{r_0'}{r_1}-\nref{r_0'}{r_2}=\nref{r_0}{r_1}-\nref{r_0}{r_2}$.
\end{theorem}

\begin{proof}
Immediate from $\log$ rules:
$-\log_2(r/r_0')=-\log_2(r/r_0)-\log_2(r_0/r_0')$.
\end{proof}

\begin{remark}
What looks like subtraction (moving entries) can be traded for a gauge shift
(moving the reference knot). This is the ``M\"obius ledger'' view: one band,
many possible centers; only differences matter.
\end{remark}

%===============================================================================
\section{Macro: Time--Topology Cosmology (TTC)}
%===============================================================================

Let $\tau(r)$ denote a \emph{time-density} (units chosen so $\tau$ is positive
and smooth). The gravitational acceleration is the Onu-gradient of $\tau$:
\begin{equation}
  \boxed{\,a(r)=-c^2\,\dv{}{r}\ln\tau(r)\,}
  \label{eq:ttc-accel}
\end{equation}
and the circular speed follows
\begin{equation}
  \boxed{\,v_c(r)=\sqrt{|a(r)|\,r}=\sqrt{c^2 r\,\abs{\dv{}{r}\ln\tau(r)}}\,}.
\end{equation}
A simple luminous+baseline model
\begin{equation}
  \tau(r)=\tau_0\!\left(1+\Bigl(\frac{r_0}{r}\Bigr)^\alpha\right),
\end{equation}
yields rising-to-flat rotation curves without invoking hidden mass: the outer
plateau tracks a constant log-slope of $\tau$.

\paragraph{Reproducible code.}
Reference implementation (Appendix~\ref{app:code}) reproduces the curves and
diagnostics from \eqref{eq:ttc-accel}.

%===============================================================================
\section{Micro: Ma'at Atomic Stability}
%===============================================================================

Disordered fields cool under the scale Laplacian on the rail
$(z\partial_z)^2=\partial_{ss}$, reducing log-variance and concentrating
spectral mass into a stable eigenchannel (the eight-lobe $Y_3^{\pm2}$ container).
We use the empirical/pedagogical predictive law
\begin{equation}
  \boxed{\,\Varlog(C)=V_0+k\,(1-C)\,}
\end{equation}
linking container concentration $C\in[0,1]$ to log-variance of observables.
Interpretation: leakage $(1-C)$ carries a variance cost; Ma'at-constrained
cooling pushes $C\to1$ and $\Varlog\to V_0$.

%===============================================================================
\section{Unification: Sliding Planck Threshold}
%===============================================================================

To connect macro and micro continuously, we let the effective Planck scale
\emph{slide} along the ledger:
\begin{equation}
  \boxed{\,\ell_{\mathrm P}(S)=\ell_{\mathrm P}^{(0)}\,2^{-S/B}\,}
  \qquad(S=\text{ledger position},\; B>0).
\end{equation}
Thus the continuum never breaks; it rescales. The Quasmos loop closes:
galactic gradients shape containers (TTC); the Onu gauge carries them down the
log-rail; the field meets a local threshold and curls into Ma'at-stable horns.

%===============================================================================
\section{Discussion and Conclusion}
%===============================================================================

The \emph{Quasmos} is not a replacement for domain physics but a unifying gauge:
place systems on the log-rail, expose their recursion parameter, and identify
where residuals are stored. Horizons, vortex cores, and ledger knots are the
same algebraic role in different clothes. With TTC and Ma'at providing macro
and micro closures, the sliding threshold gives thermo-temporal continuity.

%===============================================================================
\appendix
\section{TTC Simulation (source excerpt)}
\label{app:code}
% If the file is available at compile time, this will embed it; otherwise we show a link.
\IfFileExists{/mnt/data/ttc_sim.py}{%
  \lstinputlisting[language=Python]{/mnt/data/ttc_sim.py}
}{%
  \noindent\textit{Source file:}
  \href{sandbox:/mnt/data/ttc_sim.py}{/mnt/data/ttc_sim.py}
}

\section{Additional assets (download links)}
\vspace{-0.5em}
\begin{itemize}[leftmargin=1.5em]
  \item TTC preprint (v1): \href{sandbox:/mnt/data/TTC_v1_Preprint.pdf}{/mnt/data/TTC_v1_Preprint.pdf}
  \item Referee crib sheet: \href{sandbox:/mnt/data/TTC_v1_Referee_Crib_Sheet.pdf}{/mnt/data/TTC_v1_Referee_Crib_Sheet.pdf}
  \item Press summary: \href{sandbox:/mnt/data/TTC_v1_Press_Summary.pdf}{/mnt/data/TTC_v1_Press_Summary.pdf}
  \item CMB test script: \href{sandbox:/mnt/data/ttc_cmb_test.py}{/mnt/data/ttc_cmb_test.py}
  \item Power-spectrum amplitude: \href{sandbox:/mnt/data/ttc_pcl_amp.py}{/mnt/data/ttc_pcl_amp.py}
  \item PDP protocol code: \href{sandbox:/mnt/data/pdp_protocol_code.pdf}{/mnt/data/pdp_protocol_code.pdf}
\end{itemize}

\end{document}
