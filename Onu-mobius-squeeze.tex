% !TEX program = lualatex
\documentclass[11pt]{article}

% ---------- PACKAGES ----------
\usepackage[margin=1in]{geometry}
\usepackage{microtype}
\usepackage{amsmath,amssymb,amsthm,mathtools}
\usepackage{bm}
\usepackage{graphicx}
\usepackage[hidelinks]{hyperref}
\usepackage{enumitem}

% ---------- TITLE ----------
\title{\textbf{Onu Gauge, Hawking--Squeeze, and the M\"obius Ledger:\\
Black Holes on the Kitchen Table}}

\author{Yay / Onu Collective}
\date{\today}

% ---------- SHORTCUTS ----------
\newcommand{\dd}{\mathrm{d}}
\newcommand{\Rpos}{\mathbb{R}_{>0}}
\newcommand{\Onu}{\mathcal{O}}
\newcommand{\sL}{\mathcal{S}}

\newcommand{\nref}[2]{n_{#1}\!\left(#2\right)}

\newtheorem{definition}{Definition}
\newtheorem{theorem}{Theorem}
\newtheorem{remark}{Remark}
\newtheorem{proposition}{Proposition}

\begin{document}
\maketitle

\begin{abstract}
We show that three apparently distinct constructions share a single
mathematical spine: (i) General Relativity rewritten in terms of a
recursive compactness parameter, (ii) a logarithmic laboratory nozzle
(the ``Hawking--Squeeze''), where flow speed and pressure become
straight lines in a log gauge, and (iii) the M\"obius ``Onu ledger,''
in which all subtraction is absorbed into a gauge choice on a positive
scale coordinate.  The common structure is a log-domain recursion
parameter that linearizes dynamics and concentrates residuals at a
horizon or core.  We prove a gauge-equivalence theorem for the ledger,
derive the Hawking--Squeeze flow laws in the same gauge, and show that
Schwarzschild geometry admits an analogous horizon coordinate.  Black
holes, nozzle vortices, and positive-ledger arithmetic are thereby
unified as specializations of a single Onu recursion identity.
\end{abstract}

\tableofcontents

% ===========================================================
\section{Introduction}
% ===========================================================

Standard physical theories are usually presented in different
coordinate systems and notations: curvature tensors in General
Relativity (GR), area--velocity relations in compressible flow,
and signed real lines in elementary calculus.  The Onu Calculus
proposes that these are all shadows of a simpler structure: a
\emph{log-domain ledger} with a single recursion parameter and a
boundary where residuals accumulate.

Three ingredients recur:

\begin{enumerate}[label=(\roman*)]
    \item a positive scale variable (radius, cross-section, length),
    \item a logarithmic ledger coordinate on that scale, and
    \item a compactness-like parameter that recurses linearly in the ledger.
\end{enumerate}

In this paper we make that common structure explicit in three domains:
gravity, laboratory fluid flow, and arithmetic.

\subsection*{Why ``Onu''?}

The term \emph{Onu} is used throughout the broader Onu Calculus
framework as a label for the positive-domain, log-linear organization
of quantities.  In this note, the word carries no prior technical
baggage: it simply names (i) the ratio-based difference operation on
$\Rpos$, (ii) the associated logarithmic ledger coordinate, and
(iii) the M\"obius-band picture in which any point can serve as the
``knot'' or reference.  Readers unfamiliar with the wider framework
may safely treat ``Onu'' as shorthand for
``log-linear positive-ledger gauge.''

\subsection*{Black holes on the kitchen table}

The phrase ``black holes on the kitchen table'' is intended literally:
once the underlying recursion form is exposed, the same structural
pattern that organizes a Schwarzschild horizon also appears in a
desktop-scale nozzle experiment and in the arithmetic of positive
ratios.  The horizon, the vortex core, and the ledger knot are
different manifestations of a common log-linear recursion with a
boundary where residuals are stored.

% ===========================================================
\section{Onu Difference and Scale Ledger}
% ===========================================================

\begin{definition}[Onu difference]
For $a,b \in \Rpos$, the \emph{Onu difference} is
\begin{equation}
    a \ominus b := \frac{a}{b} \in \Rpos.
\end{equation}
Taking logarithms in any base converts this multiplicative difference
into an additive one:
\begin{equation}
    \log(a \ominus b) = \log a - \log b.
\end{equation}
\end{definition}

\begin{definition}[Scale ledger]
Fix a reference scale $r_0 \in \Rpos$.  The associated
\emph{scale-ledger coordinate} of $r \in \Rpos$ is
\begin{equation}
    \nref{r_0}{r} := -\log_2\!\left(\frac{r}{r_0}\right).
\end{equation}
The set $\sL(r_0) := \{\nref{r_0}{r} \mid r \in \Rpos\}$ is the
\emph{scale ledger} anchored at $r_0$.
\end{definition}

The key idea is that $\Rpos$ is viewed not as a signed line but as a
single positive band.  Differences are expressed as ratios, and the
logarithm provides an additive coordinate on that band.

% ===========================================================
\section{M\"obius Ledger and Gauge Equivalence}
% ===========================================================

Operationally there are two ways to ``balance'' a ledger.

\begin{itemize}
    \item[(A)] Fix the reference $r_0$ and move the entries $r \mapsto r'$.
    \item[(B)] Fix the entries and move the reference $r_0 \mapsto r_0'$.
\end{itemize}

Geometrically, we imagine the ledger as a M\"obius band and a marked
knot labeled ``Onu'' on that band.  In scenario (A), we slide entries
around the band relative to a fixed knot.  In scenario (B), we slide
the knot along the band while keeping entries fixed.  We now show that
these are equivalent at the level of ledger \emph{differences}.

\begin{theorem}[Onu M\"obius Gauge Equivalence]
\label{thm:mobius}
Let $r_0,r_0' \in \Rpos$ be two reference scales, and define
$\nref{r_0}{\cdot}$ and $\nref{r_0'}{\cdot}$ as above.
Then for every $r \in \Rpos$,
\begin{equation}
    \nref{r_0'}{r}
    = \nref{r_0}{r} + C(r_0',r_0),
\end{equation}
where the constant
\begin{equation}
    C(r_0',r_0) := \log_2\!\left(\frac{r_0'}{r_0}\right)
\end{equation}
is independent of $r$.  Consequently, for any $r_1,r_2 \in \Rpos$,
\begin{equation}
    \nref{r_0'}{r_1} - \nref{r_0'}{r_2}
    = \nref{r_0}{r_1} - \nref{r_0}{r_2}.
\end{equation}
\end{theorem}

\begin{proof}
By definition,
\begin{align}
    \nref{r_0'}{r}
      &= -\log_2\!\left(\frac{r}{r_0'}\right)
       = -\log_2\!\left(\frac{r}{r_0}\cdot\frac{r_0}{r_0'}\right) \\
      &= -\log_2\!\left(\frac{r}{r_0}\right)
         -\log_2\!\left(\frac{r_0}{r_0'}\right) \\
      &= \nref{r_0}{r}
         + \log_2\!\left(\frac{r_0'}{r_0}\right),
\end{align}
which establishes the claimed form with
$C(r_0',r_0) = \log_2(r_0'/r_0)$.  For any pair $(r_1,r_2)$ we then have
\begin{align}
    \nref{r_0'}{r_1} - \nref{r_0'}{r_2}
      &= \bigl(\nref{r_0}{r_1} + C\bigr)
       - \bigl(\nref{r_0}{r_2} + C\bigr) \\
      &= \nref{r_0}{r_1} - \nref{r_0}{r_2},
\end{align}
so all ledger differences are invariant under the change of reference.
\end{proof}

\begin{remark}
Ledger balance conditions that depend only on differences
(affine constraints) are indifferent to the choice of $r_0$.  What
looks like subtraction (moving entries) can be reinterpreted as a
change of reference (moving the knot).  The M\"obius picture expresses
this as a topological fact: one band, many possible centers.
\end{remark}

% ===========================================================
\section{GR in Onu Gauge: Recursive Compactness}
% ===========================================================

We now pass to gravity.  General Relativity in Schwarzschild form
typically uses curvature and metric functions of radius $R$.
Onu Calculus replaces curvature with a \emph{recursive compactness}
parameter that plays the same structural role across domains.

\subsection{Compactness as recursion parameter}

For a spherically symmetric mass distribution with enclosed mass
$\mathcal{M}(R)$, define
\begin{equation}
    C(R) = u(R) := \frac{2\mathcal{M}(R)}{R c^2}.
\end{equation}
In standard GR this is gravitational compactness; $u=1$ corresponds to
the Schwarzschild radius $R=r_s$.  In the Onu view, $u$ is elevated to
a universal recursion parameter.

Consider the Schwarzschild metric
\begin{equation}
    \dd s^2 = -\left(1-\frac{r_s}{R}\right)c^2\dd t^2
              + \left(1-\frac{r_s}{R}\right)^{-1}\dd R^2 + R^2\dd\Omega^2,
\end{equation}
with $r_s = 2GM/c^2$.  The problematic factors near the horizon are
powers of $1-r_s/R = 1-u$.

\subsection{Horizon-relative log coordinate}

Introduce the horizon-relative log coordinate
\begin{equation}
    s = \ln\left(\frac{R}{r_s}\right),
    \qquad R = r_s e^{s},
\end{equation}
so that $s=0$ labels the horizon, $s>0$ the exterior, and $s<0$ the
interior.

We have
\begin{equation}
    1-\frac{r_s}{R} = 1 - e^{-s},
\end{equation}
which is smooth through $s=0$.  The apparent singularity in the metric
is therefore a coordinate artifact removed by the same log gauge used
for the ledger.

Moreover, the compactness becomes
\begin{equation}
    u(s) = \frac{r_s}{R} = e^{-s},
\end{equation}
so that
\begin{equation}
    \ln u(s) = -s.
\end{equation}
In other words, the log of the recursion parameter is itself a straight
line in the horizon gauge: compactness is an exponential of the ledger
coordinate.

This is the first member of the ``Onu triad'':
\begin{center}
    \emph{gravity: } $\ln u = -s$ is linear in the log gauge.
\end{center}

% ===========================================================
\section{Hawking--Squeeze Nozzle as Onu Flow}
% ===========================================================

We now consider a laboratory flow configuration: an axisymmetric
nozzle whose radius decreases exponentially with a scale coordinate
$s$.  We show that, in the same log gauge, both flow speed and
pressure follow simple recursion laws, with a floor that plays the
role of a horizon.

\subsection{Geometry and continuity}

Let the nozzle radius be
\begin{equation}
    R(s) = R_0 e^{-\beta s},
\end{equation}
for some taper parameter $\beta>0$.  Then the cross-sectional area
satisfies
\begin{equation}
    A(s) \propto R(s)^2 \propto e^{-2\beta s}.
\end{equation}

Assume steady compressible flow with density $\rho(s)$ and speed
$v(s)$.  Continuity $\rho v A = \text{const}$ gives
\begin{equation}
    \rho(s)\,v(s)\,e^{-2\beta s} = \rho_0 v_0,
\end{equation}
where $(\rho_0,v_0)$ are values at some reference $s=0$.

\subsection{Isentropic flow and Mach number}

Assume isentropic flow of a perfect gas with ratio of specific heats
$\gamma$.  Standard compressible-flow relations yield
\begin{equation}
    \frac{p}{p_0}
      = \left(1 + \frac{\gamma-1}{2}M^2\right)^{-\gamma/(\gamma-1)},
\end{equation}
where $M=v/c$ is the local Mach number, $c$ the sound speed, and $p_0$
the reservoir pressure.  Eliminating $\rho$ via the equation of state
and continuity, one finds that $M^2$ grows approximately as
$e^{4\beta s}$ in the accelerating region of the nozzle; a simple
model captures this by
\begin{equation}
    M^2(s) \approx M_0^2 e^{4\beta s},
\end{equation}
with $M_0$ the Mach number at $s=0$.

Plugging this into the isentropic formula motivates the
\emph{Hawking--Squeeze model}:
\begin{equation}
    \frac{p(s)}{p_0}
      = \max\!\left\{\delta_p,
        \left(1 + \frac{\gamma-1}{2}M_0^2 e^{4\beta s}\right)^{-\gamma/(\gamma-1)}
        \right\},
\end{equation}
where $0<\delta_p<1$ is a \emph{pressure floor} associated with a
vortex core at the nozzle exit.

\subsection{Log-slope law and pressure floor}

In the accelerating region (before the floor is hit), continuity plus
the approximate $M^2(s)$ behavior yield
\begin{equation}
    v(s) \propto e^{2\beta s}.
\end{equation}
Taking logarithms,
\begin{equation}
    \ln v(s) = \ln v_0 + 2\beta s.
\end{equation}
Thus a plot of $\ln v$ versus $s$ (or equivalently $\ln v$ versus
$\ln R$) is a straight line of slope $2\beta$ set purely by the
geometry of the taper.  This is the \emph{log-slope law}.

As $s$ increases and $M^2(s)$ becomes large, the isentropic factor
drives $p(s)/p_0$ down until it saturates at $\delta_p$:
\begin{equation}
    \lim_{s\to\infty}\frac{p(s)}{p_0} = \delta_p.
\end{equation}
Further increase in drive power does not appreciably lower the static
pressure; instead, the residual is carried by the vortex core.  This
is the \emph{pressure floor}.

In summary:

\begin{center}
\begin{tabular}{l}
    \emph{nozzle (Hawking--Squeeze):} \\[4pt]
    $\ln v(s) = \ln v_0 + 2\beta s$ (log-slope law),\\[2pt]
    $p(s)/p_0 \to \delta_p$ (pressure floor).
\end{tabular}
\end{center}

The flow inherits the same log-linear recursion structure seen in the
gravity case, with $\beta$ playing the role of a slope parameter and
$\delta_p$ acting as a horizon-like floor.

% ===========================================================
\section{Onu Triad: Gravity, Nozzle, Ledger}
% ===========================================================

We can now state the structural equivalence more formally.

\begin{definition}[Onu recursion form]
A system is said to be in \emph{Onu recursion form} if there exists a
scale coordinate $s$ and observable $X(s)$ such that
\begin{equation}
    \ln X(s) = \ln X_0 + \kappa s,
\end{equation}
for some constant slope $\kappa$, possibly up to a floor or ceiling
where the recursion terminates and residuals are carried by a boundary
(horizon, core, or ledger knot).
\end{definition}

\begin{proposition}[Onu triad]
The following are all instances of Onu recursion form:

\begin{enumerate}[label=(\alph*)]
    \item \textbf{Gravity.} With $s=\ln(R/r_s)$, the compactness
    satisfies $\ln u(s) = -s$.
    \item \textbf{Nozzle (Hawking--Squeeze).} With $s$ the taper
    coordinate, the flow speed satisfies $\ln v(s) = \ln v_0 + 2\beta s$,
    and $p(s)/p_0$ approaches a floor $\delta_p$.
    \item \textbf{Ledger.} With $s$ replaced by the scale ledger
    coordinate $\nref{r_0}{r}$, any ratio $X(r)/X(r_0)$ satisfying
    a multiplicative law $X\propto r^{-\alpha}$ obeys
    \begin{equation}
        \ln X(r) = \ln X(r_0) + (\alpha \ln 2)\,\nref{r_0}{r}.
    \end{equation}
\end{enumerate}
\end{proposition}

\begin{proof}
Item (a) follows from $u(r_s e^{s}) = e^{-s}$ as shown above.
Item (b) follows from the continuity and isentropic arguments leading
to $v(s)\propto e^{2\beta s}$, with the floor given by the limiting
behavior of $p(s)/p_0$.  For item (c), if $X(r)\propto r^{-\alpha}$,
then $X(r)/X(r_0) = (r/r_0)^{-\alpha}$, so
\begin{equation}
    \ln X(r)
      = \ln X(r_0) -\alpha\ln\!\left(\frac{r}{r_0}\right)
      = \ln X(r_0) + (\alpha\ln 2)\,\nref{r_0}{r}.
\end{equation}
\end{proof}

\begin{remark}
In each case, the log of the central quantity of interest (compactness,
speed, or field) is an affine function of a log-gauge coordinate.
Residuals that do not fit the simple recursion---mass beyond the
horizon, vorticity in the nozzle core, or off-ledger imbalances---are
concentrated at a boundary.  This is the sense in which ``black
holes are on the kitchen table'': the nozzle plus ledger implement the
same recursion pattern as a horizon.
\end{remark}

% ===========================================================
\section{Broader Context}
% ===========================================================

The Onu recursion pattern resonates with several well-known structures
in mathematical physics:

\begin{itemize}
    \item \textbf{Renormalization group (RG).}
    Many RG flows become linear in a logarithmic length scale,
    $t = \ln(\mu/\mu_0)$, with couplings following $\dd g/\dd t =
    \beta(g)$.  Near fixed points, $\beta(g)\approx \lambda(g-g_\ast)$
    yields log-linear scaling laws.  The Onu ledger can be viewed as
    an RG-like flow on a positive scale band with a single effective
    parameter.

    \item \textbf{Entropy and area laws.}
    Black-hole entropy scaling with horizon area, and various
    entanglement entropy area laws, can be interpreted as ledger
    statements about how information density is organized across
    scales.  Compactness $u(R)$ acts as an Onu-style recursion of
    ``information per radius''.

    \item \textbf{Diffusion in log space.}
    Diffusion processes on multiplicative variables (e.g.\ log-normal
    distributions) become ordinary Brownian motion in the log
    coordinate.  This is closely related to the ledger diffusion
    picture in other Onu work, where information density diffuses on
    the scale ledger.
\end{itemize}

These analogies suggest that the Onu gauge is not an isolated trick
but part of a broader family of log-linear recursion frameworks.

% ===========================================================
\section{Kitchen-Table Hawking--Squeeze Experiment}
% ===========================================================

To emphasize testability, we outline a concrete experiment capable of
probing the Onu structure in a tabletop setting.

\subsection{Geometry and setup}

\begin{itemize}
    \item Construct a smooth axisymmetric nozzle with radius
    \[
      R(s) = R_0 e^{-\beta s}, \quad s\in[0,s_{\max}],
    \]
    where $s$ is measured along the centerline and $\beta$ is set by
    the machining taper.  3D printing or CNC machining allow accurate
    logarithmic profiles.
    \item Drive a compressible gas (e.g.\ air) from a plenum at
    pressure $p_0$ through the nozzle into ambient conditions.
    \item Instrument the nozzle with:
        \begin{enumerate}
            \item wall pressure taps at a sequence of $s_i$,
            \item at least one Pitot tube or equivalent to measure
            local flow speed or Mach number,
            \item a microphone or flow-tone sensor to track frequency
            shifts of an injected acoustic tone.
        \end{enumerate}
\end{itemize}

\subsection{Predicted signatures}

The Hawking--Squeeze model predicts two primary signatures:

\begin{enumerate}[label=(\roman*)]
    \item \textbf{Log-slope law.}
    Plot $\ln v(s_i)$ versus $s_i$ (or equivalently
    $\ln v$ versus $\ln R$).  The data should align along a straight
    line of slope $2\beta$ set purely by the taper geometry.  The same
    slope should appear in the log of the frequency up-shift of a tone
    passed through the nozzle.

    \item \textbf{Pressure floor.}
    Plot $p(s_i)/p_0$ versus $s_i$ for increasing drive amplitude.
    Curves should converge to a common asymptotic profile with a floor
    value $\delta_p$ at large $s_i$, independent of further increases
    in drive once a stable vortex core is established at the exit.
\end{enumerate}

A minimal figure for this experiment would include:

\begin{itemize}
    \item a schematic of the nozzle geometry with logarithmic $R(s)$,
    \item a log-linear plot of $\ln v$ versus $s$ showing the fitted
    slope $2\beta$,
    \item pressure profiles illustrating convergence to the floor
    $\delta_p$ as drive increases.
\end{itemize}

Successful observation of both signatures would provide an
operational realization of the Onu recursion form and an accessible
analog of horizon physics.

% ===========================================================
\section{Discussion and Outlook}
% ===========================================================

We have shown that a single log-domain recursion pattern---the Onu
gauge---underlies three seemingly disparate phenomena: Schwarzschild
compactness, logarithmic nozzle flow, and positive-ledger arithmetic.
The M\"obius gauge-equivalence theorem (Theorem~\ref{thm:mobius})
demonstrates that subtraction can be absorbed into a choice of
reference; the gravity and nozzle examples show that horizons and
vortex cores play the same structural role as ledger knots.

Beyond the immediate triad, at least three directions suggest
themselves:

\begin{enumerate}
    \item \textbf{Experimental tests.}
    The kitchen-table Hawking--Squeeze experiment outlined above
    provides falsifiable predictions: a geometric log-slope law and a
    drive-independent pressure floor.  Variants could explore
    different tapers, working fluids, or acoustic diagnostics.

    \item \textbf{Further domains.}
    The same log-gauge may organize:
    \begin{itemize}
        \item cascades in turbulence (energy flux across scales),
        \item hierarchical networks (degree distributions in log space),
        \item multiplicative finance processes (volatility on log-returns),
        \item information cascades in communication systems.
    \end{itemize}
    Each case invites a reformulation in terms of an Onu ledger with a
    compactness-like recursion parameter.

    \item \textbf{Universality conjecture.}
    Informally, we conjecture that any system with:
    \begin{enumerate}
        \item a positive scale variable,
        \item a distinguished boundary (horizon, core, or sink), and
        \item a single effective recursion parameter,
    \end{enumerate}
    can be mapped into Onu recursion form by an appropriate log gauge.
    Classifying such universality classes and their invariants is a
    natural direction for further work.
\end{enumerate}

In this sense, the Onu gauge is not a new physical theory but a
unifying pedagogical and mathematical framework: a way of placing
gravity, vortex flows, and arithmetic on the same positive, twisted
band.

% ===========================================================
\section*{Figures}
% ===========================================================

\begin{figure}[h]
    \centering
    % Replace with your vortex-core artwork or data-driven schematic
    \includegraphics[width=0.8\textwidth]{fig/hawking_squeeze_vortex.png}
    \caption{\textbf{Hawking--Squeeze nozzle in Onu gauge.}
    A logarithmic taper straightens the flow laws in the log coordinate
    $s$, with velocity following a line of slope $2\beta$ and pressure
    falling to a vortex-core floor $\delta_p$.  In a data-driven
    version of this figure, axes would be labeled in physical units
    and straight-line fits overlaid on measured points.}
\end{figure}

\begin{figure}[h]
    \centering
    % Replace with your "Two Ways to Onu" slide filename
    \includegraphics[width=0.8\textwidth]{fig/two_ways_to_onu.png}
    \caption{\textbf{Two Ways to Onu.}
    Left: flow the M\"obius ledger through a fixed Onu knot (move
    entries).  Right: move the Onu knot along a fixed ledger (change
    reference).  The gauge-equivalence theorem shows these procedures
    are mathematically identical at the level of ledger differences.}
\end{figure}

\end{document}
